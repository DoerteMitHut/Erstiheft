%==============Directories=============================
\def \imagePath{./images}% directory containing images
%==============title image=============================
\def \titleImageFilename{brinpage}% filename of title image without suffix
\def \titleImageDescription{Sergey Brin und Larry Page, Gründer von Google}
%==============space time coordinates==================
%==locations==
\def \notYetSet{[$\blacksquare\blacksquare\blacksquare$]\ }
\def \roomFreshmenBrunch{UFO 0/02}                                  % Raum für's Erstifrühstück
\def \initialSegregationMeetingPoint{in der Eingangshalle des Gebäudes NA (Ebene 01)}                    % Treffpunkt Tutorieneinteilung
\def \roomIntroductoryEvent{\notYetSet}                             % Raum der Einführungsveranstaltung
\def \PubcrawlMeetingPointOne{\notYetSet}                         % Erster Treffpunkt Kneipentour
\def \PubcrawlMeetingPointTwo{Am Bochumer Hauptbahnhof (Haupthalle)}% Zweiter Treffpunkt Kneipentour
\def \PubcrawlStartingPub{Game am Bochumer Rathaus}                 % Erste Kneipentourkneipe
\def \excursionDestination{in ein Gruppenhaus in der Gegend des Nürburgrings }                              % Ziel der Erstifahrt im Akkusativ für Zielangebende Präpositionen
%==dates and times== 
\def \currentTwoDigitYear{\the\numexpr(\the\year) - 2000\relax}
\def \firstDayOfFreshmenWeek{08.10.}
\def \lastDayOfFreshmenWeek{12.10.}

\def \initialSegregationMeetingTime{08:00}
\def \AudimaxAddressingStartingTime{09:30}
\def \startingTimeOfTutorProgram{10:00}
\def \startingTimeOfIntroductoryEvent{14:15}
\def \firstMeetingTimePubCrawl{17:00}
\def \secondMeetingTimePubCrawl{17:30}
\def \startingTimeFreshmenBrunch{10:00}

\def \lastDayOfExcursion {11.11.}

\def \firstDayOfExcursion {09.11.}



Um dir den Einstieg in den Unialltag zu erleichtern, beginnt die erste Woche der Vorlesungszeit nicht direkt mit dem vollen Lernprogramm.

In der Ersti-Woche vom \makeDate{\firstDayOfFreshmenweek} bis zum \makeDate{\lastDayOfFreshmenweek} hast du die Gelegenheit, auf einfache Weise Leute aus deinem Studiengang kennen zu lernen. Schließlich wirst du mit diesen einen mehr oder minder großen Teil deines Studiums verbringen und zusammen macht es einfach mehr Spaß. Die Ersti-Woche dient auch dazu, dir ein paar Einblicke in das Uni-Leben zu geben und dir Dinge zu zeigen, auf die du sonst vielleicht gar nicht so ohne weiteres gestoßen wärest.\\

Für dich beginnt die Woche am Montag um \initialSegregationMeetingTime Uhr \initialSegregationMeetingPoint . Dort gibt es eine kurze Begrüßung und anschließend stellen sich die Tutoren vor. Diese teilen euch dann in Gruppen ein, die für die Dauer des ersten Semesters beibehalten werden sollten. Um \AudimaxAddressingStartingTime Uhr geht es dann weiter mit der zentralen Immatrikulationsfeier im Audimax. Dort versammeln sich alle Erstis der RUB und lauschen andächtig den Ansprachen von Rektor, Bürgermeister und Anderen um dann im Anschluss vom Fachschaftsrat auf dem Forum begrüßt zu werden.\\
Am Dienstag geht es um \startingTimeOfTutorProgram Uhr mit dem Tutorenprogramm weiter. Um \startingTimeOfOfficialWelcomingEvent Uhr beginnt die offizielle Einführungsveranstaltung für den Bachelor-/Master- Studiengang „Angewandte Informatik“ im \roomIntroductoryEvent. Noch wichtiger: Am Abend veranstalten wir eine gemeinsame Kneipentour in die Bochumer Innenstadt. Hier ist um \firstMeetingTimePubCrawl Uhr erster Treffpunkt \PubcrawlMeetingPointOne mit Wegbier, zweiter Treffpunkt um \secondMeetingTimePubCrawl Uhr \PubcrawlMeetingPointTwo und ab \startingTimePubCrawl Uhr beginnt \PubcrawlStartingPub die Kneipentour.\\

Am Mittwoch ist dann der Tag der Fachschaft und der Fachschaftsrat darf in Aktion treten. In \roomFreshmenBrunch werden wir zusammen ab \startingTimeFreshmenBrunch Uhr brunchen und uns vorstellen, damit du weißt, wen du bei allen auftretenden Fragen und Problemen ansprechen kannst. Wenn du Fragen – egal, welcher Art auch immer – haben solltest, zögere nicht sie auszusprechen, dafür sind wir ja schließlich da. Wir planen außerdem eine Campusrallye, die dir die wunderschönen Betonklötze näher bringen soll, die du die nächste Zeit täglich besuchen darfst. Die Gewinner erhalten tolle Geschenke.\\

Ab Freitag erwarten dich dann die ersten Vorlesungen und das Studieren geht richtig los! Um dich nach den ersten Wochen von dem ersten Schock zu erholen, laden wir dich ein mit uns gemeinsam \excursionDestination zu fahren. Dort wollen wir das Wochenende (\makeDate{\firstDayOfExcursion} - \makeDate{\lastDayOfExcursion}) ganz locker und vor allem mit Spaß genießen. Mehr Infos dazu & Anmeldemöglichkeiten gibt es schon am Tag der Fachschaft.\\

Wie du siehst, erwartet dich ein volles Programm. Aus Erfahrung lässt sich jedem Erstsemester nur raten, die Termine wahrzunehmen, um Kontakte zu knüpfen so wie Uni und Fachschaft kennen zu lernen.\\

Also, man sieht sich!