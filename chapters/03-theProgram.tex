Im Jahre 2004 entschied man sich an der RUB, auf den IT-Zug aufzuspringen. Dabei wollte man den Spagat zwischen der eher theoretischen Informatik, wie sie schon länger als Teil der Mathematik existiert, und den vielen Anwendungsfächern sowie Grundlagen aus anderen Fachbereichen, die einem nicht fehlen sollten, wagen. Diese breite Fächerbasis fasste man medienwirksam unter dem Begriff „polydisziplinär“ zusammen. 2013 dann ging die Leitung des Studiengangs von der Fakultät für Elektro- und Informationstechnik (ET/IT) auf das Institut für Neuroinformatik (INI) über, das viele Lehrveranstaltungen für den Studiengang anbietet.

In den ersten Semestern wirst du eine ganze Menge Grundlagen der Mathematik und Informatik kennenlernen. Aller Anfang ist schwer und so lehrt die Erfahrung, dass in der AI - wie in den meisten Studienfächern - die Zahl deiner Kommilitoninnen und Kommilitonen innerhalb der ersten Semester stark schrumpfen wird. Dabei sollten sich alle Zweifelnden bewusst sein, dass die oft trockenen Vorlesungen des Kernbereichs nach dem dritten Semester abnehmen und durch Veranstaltungen der selbst gewählten Vertiefungsfächer aus zahlreichen Wahlkatalogen ergänzt werden.

Solltest du zu irgendeinem Zeitpunkt Fragen zu deinem Studium haben, zögere nicht, deinen Tutor oder ein Mitglied des Fachschaftsrates anzusprechen und um Beistand und Information zu ersuchen.

Die Sitzungen des Fachschaftsrates finden im Abstand von einer Woche das ganze Semester über statt.