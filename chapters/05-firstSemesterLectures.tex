Höhere Mathematik I
Für manche AI-ler ist dies die schwerste Vorlesung - andere hingegen haben damit weniger Probleme. Fakt ist, dass es sich bei Frau Kasco um eine sehr nette und hilfsbereite Dozentin handelt, die ihren Studenten gerne entgegenkommt. Gegen Ende des Semesters wird es zwei Probeklausuren geben, welche dir eine gute Gelegenheit geben, zu prüfen, wie gut du auf die Klausur vorbereitet bist. Da du dir in jeder Probeklausur bis zu 10 Bonuspunkte verdienen kannst, die sogar vor Bestehen der echten Klausur auf dein Endergebnis angerechnet werden, solltest du dich in jedem Fall gut auf die Probeklausuren vorbereiten! Du benötigst im Extremfall also bloß 30% in der Klausur, um die Prüfung zu bestehen.

Hinweis: Du darfst 5 beidseitig handbeschriebene Zettel in die Klausur mitnehmen.
Statistik
Lass dich nicht von dem einfachen Anfang dieser Vorlesung täuschen, denn Statistik wird mit der Zeit ziemlich verwirrend. Es mag zu Beginn noch einfach sein – Klassische Wahrscheinlichkeitsrechnung und Kombinatorik – später kommen Bernoulli-Verteilung, Ungleichung von Chebyshev, Signifikanztests... wer da nicht am Ball bleibt, weiß bald nicht mehr, wo oben und unten ist.

Hinweis: Formelsammlung und Skript werden vom Lehrstuhl verkauft (und online gestellt). In der Regel werden sie auch in der ersten Vorlesung angeboten, also sollest du Geld dabei haben und dir damit den Weg zum Lehrstuhl sparen. In der Klausur ist nur die offizielle (und ausgedruckte) Formelsammlung erlaubt.

Tipp: Statistik ist sehr unintuitiv, die offensichtliche Lösung ist oft falsch und die richtige Lösung manchmal nicht logisch nachvollziehbar. Also besser 2x nachdenken und im Zweifelsfall stur durchrechnen.

Informatik I
Während sich bis zum Wintersemester 14/15 der berühmtberüchtigte Professor Helmut Balzert die Ehre gab, die Erstsemester in den Künste der Programmierung zu unterweisen, übernimmt dies seit dem Wintersemester 15/16 Professor Hübner. Da der Lehrstuhl von Professor Hübner eher hardwareorientiert und maschinennah arbeitet, wird in dieser Veranstaltung die Programmiersprache C++ gelehrt, welche aufgrund ihrer manchmal schwer nachvollziehbaren Eigenheiten von Einsteigern oft als eher schwierig empfunden wird. 

Programmieren in C
Allen, die schon programmieren können, dürfte die Veranstaltung anfangs recht simpel erscheinen. Es geht um Datentypen, if- und while-Blöcke etc. Allerspätestens bei Arrays und Zeigern sollten aber alle aufpassen, denn hier haben viele Leute Probleme und C hat da so seine Eigenarten.

Tipp: Die Vorlesung wie auch die Klausur sind eher theoretisch ausgerichtet. Es genügt also nicht, programmieren zu können; man muss auch die Details von C kennen und genau wissen, wie die Sachen unter der Haube funktionieren.

Wirtschaftlichkeitsanalyse
Hier bewegst du dich im Bereich der BWL. Frau Wischermann wird sich nach Kräften bemühen, euch den Sinn der Kosten- und Investitionsrechnung ersichtlich zu machen. Wenn du Fragen hast, gibt es keinen Grund, sie nicht zu stellen. Sie wird gerne Sachen wiederholen, bis sie wirklich alle verstanden haben.

Tipp: Die vorgestellten Verfahren werden in der Klausur in kurzer Zeit abgefragt. Eine gewisse Routine ist von Vorteil und Aufgabenstellungen sollten genauestens gelesen werden. Im Zweifelsfall nachfragen.

Nichttechnisches Wahlfach:
Für das erste Semester ist neben den bereits genannten Fächern das so genannte Nichttechnische Wahlfach vorgesehen. Hier sollst du ein Fach belegen, mit dem du über den Tellerrand hinaus schaust. Auf der Webseite der AI findet sich eine (nicht abschließende!) Liste mit möglichen Fächern. Bei anderen Fächern solltest du bei der Studienberatung nachfragen. Mit Sprachen (nicht Programmiersprachen!) oder Sport bist du aber auf der sicheren Seite. Als Empfehlung können wir "Englisch für Angewandte Informatiker" nennen, welches genau die 5 CP gibt, die du für das Nichttechnische Wahlfach brauchst. Du kannst übrigens auch mehr als ein Fach wählen, falls du mit einem alleine nicht auf 5 CP kommst.

Hinweis: Das Nichttechnische Wahlfach hat seinen Weg durch eine Vorschrift ins erste Semester gefunden, die für Studienpläne ziemlich penibel 
30 CP/Semester vorschreibt. Das Nichttechnische Wahlfach ist erfahrungsgemäß in späteren Semestern besser angesiedelt, wo bereits Abweichungen vom Musterstudium erfolgt sind und sich vielleicht Lücken auftun, die es effizient zu schließen gilt. 