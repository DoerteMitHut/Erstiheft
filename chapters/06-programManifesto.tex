Allgemeines
Als Student studiert man unter einer Prüfungsordnung (PO). Diese legt den Aufbau des Studiums und die „Spielregeln“ fest. Jeder, der sich ab dem WS13/14 in die Angewandte Informatik einschreibt, studiert unter der PO13. Die Grundzüge wollen wir dir hier etwas näher bringen, garantiert präzise Informationen gibt es aber nur im Originaltext der PO.
www.ai.rub.de/studierende/ordnungen/po13.html

Eckdaten und Credit Points
Die Regelstudienzeit beträgt 6 Semester, die Erfahrung zeigt jedoch, dass nur Wenige diesen strikten Zeitplan einhalten können. Ein oder zwei Semester an den Bachelor anzuhängen, ist jedoch kein Weltuntergang, da man sowohl zum Winter- als auch zum Sommersemester in den Master eingesteigen kann. In dieser Zeit gilt es, mindestens 180 Credit Points (CP) zu sammeln, wobei ein CP – so die Idee – etwa 30 Arbeitsstunden im Semester entsprechen sollte, wobei darin sowohl Anwesenheit in Vorlesungen als auch das Pauken zu Hause enthalten sind.
Damit kannst du dir leicht ausrechnen, dass du in jedem Semester ca. 30 CP erlangen solltest und dementsprechend ist das Studium auch aufgebaut. Diese Rechnung geht davon aus, dass du immer alles bestehst, was oft nicht der Fall ist.

Prüfungen
CP sammelst du durch das Bestehen von Veranstaltungen, aus denen sich das Studium zusammensetzt. Eine Veranstaltung besteht in den meisten Fällen aus wöchentlichen Vorlesungen und Übungen und deckt einen Themenkomplex ab. In der Vorlesung werden dabei Verfahren und Methoden vorgestellt und erklärt, die du dann in der Übung bzw. beim Pauken selbstständig anwenden sollst.
Für jede Veranstaltung, die du bestehen willst, musst du eine Prüfungsleistung erbringen. Dies ist in der Regel eine Klausur am Ende des Semesters, manchmal aber auch eine mündliche Prüfung, eine Präsentation oder die Abgabe von Aufgaben während der Vorlesungszeit. Die meisten Fächer im Studium werden benotet, und zwar mit Prozentpunkten zwischen 0 und 100, wobei 50% für ein Bestehen erforderlich sind. Bei reinen Multiple-Choice Klausuren können abweichende Kriterien anfallen. Bei allen anderen Fächern genügt es, sie mit mindestens 50% zu bestehen. Hier spricht man von einem Leistungsnachweis.
Klausuren sollen nach 4 Wochen bewertet sein, worauf man allerdings manchmal vergebens hofft. Wartezeiten von mehreren Monaten kommen manchmal vor. Des Weiteren hast du das Recht auf eine Einsicht, d.h. nachdem die Klausur bewertet wurde,  kannst du sie noch mal ansehen und auf eventuelle Mängel in der Bewertung hinweisen. Auch hier steht die Realität noch weit hinter der Idee zurück, manche Dozenten bieten gar keinen Termin für die Einsicht an und lassen dich nur nach Terminvereinbarung die Klausur einsehen.

Nichtbestehen und Rauswurf
Nicht bestandene Prüfungen müssen wiederholt werden, wobei (fast) jede Prüfung 2x im Jahr angeboten wird. Du solltest es aber gar nicht erst dazu kommen lassen, denn du hast - im Gegensatz zur PO09 - pro Prüfung nur 3 Versuche, also 2 Wiederholungsversuche. Fällst du also einmal durch, sollte dich das nicht direkt entmutigen, aber halte dich ran, denn sonst hast du im nächsten Semester noch mehr Arbeit. Wenn du 3x durchfällst, wird dir die Uni ein „endgültiges Nichtbestehen“ (ENB) bescheinigen, was das Ende deines Studiums (und i.d.R. aller anderer Informatik-verwandter Studiengänge) bedeutet.

Verbesserungsversuche
Deine Note passt dir nicht? Im Vergleich zu deinen Vorgängern bist du da in einer besseren Lage: Seit dem WS 15/16 darfst du nicht nur für drei, sondern für fünf Prüfungen einen Verbesserungsversuch in Angriff nehmen, wobei das beste Ergebnis zählt. Das gilt nur für bestande Prüfungen. Um das 3-Versuche-Limit kommst du damit also nicht herum!

An- und Abmeldung zu Prüfungen
Zu Prüfungen muss man angemeldet sein! Im ersten und zweiten Semester wirst du für alle Pflichtprüfungen automatisch angemeldet. Für Wahlpflichtfächer und das Nichttechnische Wahlfach musst du dich eigenständig anmelden. Das Gleiche gilt für Pflichtfächer ab dem dritten Semester. Dies tust du entweder beim Prüfungsamt oder online über die Plattform FlexNow. Bist du für eine Prüfung angemeldet und bestehst sie nicht, wirst du automatisch für die Wiederholungsprüfung zum nächstmöglichen Termin angemeldet.
Von Prüfungen kann man sich auch abmelden, allerdings nicht im ersten Semester. Im Zweiten ist ein Beratungsgespräch dafür erforderlich. Danach kannst du dich selbstständig abmelden. Allerdings ist dies für jede Prüfung nur 3x möglich. Für den nächstmöglichen Termin wirst du dann wieder angemeldet. Außerdem kannst du bis zu drei von dir gewählte Vertiefungsmodule durch andere Vertiefungsmodule austauschen, allerdings nicht nachdem du endgültig nicht bestanden hast (3x durchgefallen) und auch nicht, nachdem du die neue Prüfung bereits geschrieben hast. Solltest du krankheitsbedingt nicht zu einer Prüfung erscheinen können, ist ein ärztliches Attest erforderlich, damit dir kein Versuch abgezogen wird!

Vertiefungsbereich und Wahlfächer
Darüber, welche Module du bestehen musst, gibt das Modulhandbuch Aufschluss. Zusätzlich zu den Pflichtmodulen gibt es noch den Vertiefungsbereich, der mit Modulen im Gesamtumfang von 30 CP gefüllt werden will. Hierzu kannst du Veranstaltungen aus mehreren Katalogen frei wählen. Die verschiedenen Veranstaltungen solltest du im Zeitraum vom 3. bis zum 6. Semester belegen. Vorher gibt es aber noch die „Nichttechnischen Wahlfächer“, die du namensgerecht frei belegst, sie sollten zusammen 5 CP umfassen und  theoretisch im 1. Semester absolviert werden.  Allerdings kommen die Wenigsten im 1.  Semester dazu.
Später im Studium hast du dann noch ein Studienprojekt, ein Seminar und am Ende die Bachelorarbeit hinter dich zu bringen.

Dein Abschluss
Wenn du nun alle Module bestanden und 180 CP erreicht hast, erhältst du ein Bachelorzeugnis, in dem deine Durchschnittsnote, die Noten der einzelnen Module und sowohl das Thema als auch die Note deiner Bachelorarbeit festgehalten werden. Ferner bekommst du eine Bachelorurkunde und darfst dich „Bachelor of Science“ und „Ingenieur“ nennen.
Ab in den Master!
Lust auf mehr? Für 120 weitere CP gibt's den „Master of Science“, der auf 4 Semester ausgelegt ist. Bis dahin ist es noch ein weiter Weg. Doch bereits jetzt solltest du wissen, dass die Masterstudienplätze nach einem NC-Verfahren vergeben werden. Es ist also sehr ratsam, sich eine gute Durchschnittsnote im Bachelor zu erarbeiten, da man so die Chancen auf einen Masterstudienplatz maximiert.