Dieses Kapitel richtet sich hauptsächlich an die Masterstudenten, die ihren Bachelor woanders gemacht haben, aber vielleicht finden auch die bisherigen AI-ler ein paar nützliche Informationen. Abgesehen von den bachelorspezifischen Sachen in diesem Erstiheft sind die meisten Informationen auch für Masterstudenten nützlich. Deswegen folgt hier eine kleine Zusammenfassung der verschiedenen Studienmöglichkeiten, die man hier als Masterstudent hat.

Pflichtveranstaltungen
Da wäre zunächst das Fehlen von Pflichveranstaltungen. In früheren Prüfungsordnungen gab es davon nur zwei, jetzt kann man frei wählen, auch wenn im Wahlpflichtmodul (20CP) die Auswahl etwas enger ist. Welche Veranstaltungen man aus diesem Modul nimmt, ist natürlich Geschmackssache. Wer an Theoretischer Informatik Spaß hatte, dem dürfte Komplexitätstheorie sicher gefallen. Effiziente Algorithmen wird von Frau Kacso gehalten, die keinen Studenten sitzen lässt und auch gerne in ihrer Sprechstunde alles lang und breit noch einmal erklärt.
Groupware und Wissensmanagement ist eine ganz andere Art von Vorlesung, das geht schon viel mehr in Richtung Geisteswissenschaften. Wenn man keine Klausuren mag, ist man hier richtig, denn benotet wird das Fach nach mehreren Projektaufgaben, die man im Team bearbeitet.
Supervised- und Unsupervised Methods wechseln schonmal das Vortrags- und Prüfungsformat,  was allerdings dem Inhalt keinen Abbruch tut. Die Materie wird systematisch und nachvollziehbar erarbeitet und beide Dozenten verstehen sich darauf, auf den einzelnen einzugehen und Fragen zu beantworten.
Parallel Computing und Nebenläufige Programmierung behandeln ähnliche Inhalte und prüfen im Falle des ersteren im Rahmen einer Projektarbeit , im Falle des letzteren im Rahmen einer Klausur. Nebenläufige Programmierung erntet häufig Kritik am Dozenten, soll Inhaltlich jedoch recht lohnenswert sein, wenn man sich penibel an die Vorgaben aus der Vorlesung hält. 



Vertiefungsmodule
Die Vertiefunsgmodule sind die eigentlich interessanten Fächer. Hier kann man sich aus einem Katalog zwischen einigen verschiedenen Bereichen Vorlesungen aussuchen.

Ingenieurinformatik
In der Ingenieurinformatik findet man unter anderem Fächer, die den Bauingenieuren nahe liegen, aber auch allgemeinere Fächer wie Product Lifecycle Management oder IT im Engineering.

Neuroinformatik
Die Neuroinformatikfächer beschäftigen sich mit maschinellen Lernverfahren, Computersehen und autonomer Robotik. Die Kurse hier sind meist recht mathematiklastig und die Gruppen dementsprechend recht klein, aber sie lohnen sich wirklich, und die Neuroinformatiker bieten meist sehr interessante Masterarbeiten und Studienprojekte an. Lasst euch von Professor Wiskotts teilweise wirklich schwierigen Aufgabenzetteln nicht erschrecken, denn die Prüfungen sind im Verhältnis zu den Übungsaufgaben recht leicht. Professor Wiskott geht auch gerne mit den Studenten nach der Vorlesung in der Mensa essen, das sollte man sich nicht entgehen lassen.

Kryptologie und TI
Zu Kryptologie und Theoretische lässt sich sagen, dass es sehr mathematiklastig ist. Das bedeutet hier zum Teil weniger Rechnen als Beweisen. Trotzdem (oder gerade deswegen) können diese Vorlesungen wirklich interessant werden, und um Mathematik kommt man als Informatiker sowieso nicht herum, zumindest nicht, wenn man es richtig macht.

Operations Research & Management
Operations Research... ist BWL. BWL-Bashing ist unter nicht wenigen [AI]-lern so eine Art Freizeitvergnügen: wenn man nichts besseres zu bereden hat, lästert man über die "BWLer". Wenn euch die Vorlesung anspricht, oder euch Einführung Management Science gefallen hat, dann lasst euch dennoch nicht davon abhalten, sie zu hören.

Bioinformatik
Die Bioinformatik-Sektion ist neu im Modulhandbuch und es existieren entsprechend spärliche Erfahrungswerte. Probiert's aus und sagt uns Bescheid.

Weitere Veranstaltungen
Bleiben noch die Seminare (sucht euch irgendwas Interessantes aus! Man kann auch Seminare machen, die nicht in der Liste stehen, solange diese vom Thema her passen), die Freien Wahlfächer (irgendwas, was man schon immer mal machen wollte, von Astrophysik bis Zahnmedizin), das Studienprojekt (im Prinzip wie im Bachelor, nur vom Umfang und Anforderung höher) und die Masterarbeit.

Für Letztere sollte man viel Zeit einplanen, normalerweise ist sie die einzige Veranstaltung, die man im Semester hat. Überlegt euch am besten schon früh, an welchem Lehrstuhl oder zu welchem Thema ihr die Arbeit machen wollt, fragt rechtzeitig nach und lernt die Leute ein bisschen kennen, die dort am Lehrstuhl arbeiten.

Das war es zu den Wahlmöglichkeiten. Ansonsten kann man hier eigentlich nur die Tipps für Bachelor wiederholen. Lernt Leute kennen, knüpft Kontakte, seid sozial (auch wenn sich das als Klischeeinformatiker vielleicht komisch anhört), denn es hilft immer, ein paar Leute zu kennen. Man kann zusammen lernen, sich gegenseitig mit Vorlesungsmitschriften aushelfen, einander an wichtige Termine erinnern, oh, und natürlich auch nicht-Uni-Kram machen.