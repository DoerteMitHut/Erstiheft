Unsere Fachschaft betreibt eine Dreifaltigkeit aus Blog, Forum und Cloud, die dem Austausch unter den Studenten dient. in der Cloud finden sich einige Materialien zu verschiedenen Veranstaltungen und der Fachschaftsrat freut sich auch über eure Beiträge und Mitschriften.
	

  Bilde mit anderen Studenten Lerngruppen. Sich gemeinsam durch schwierige Themen zu kämpfen hilft und stärkt (oder verdirbt) den Charakter. Hier sammelst du bereits erste Erfahrungen für spätere Gruppenarbeiten wie z.B. das Studienprojekt. Aber vergiss nicht: nur weil du in einer Lerngruppe bist, heißt es nicht, dass du dich von den Anderen abgrenzen solltest, denn das Studium verläuft meist nicht geradlinig und so kann es schnell passieren, dass du im vierten Semester in einigen Fächern alleine sitzt.

Den Bachelor in 6 Semestern zu machen ist machbar aber stressig. Wenn das BAföG nichtmer fließt, lässt sich als Informatiker prima nebenher arbeiten und  ein, zwei Semester mehr zu brauchen ist weit verbreitet. 
5 Verbesserungsversuche ermöglichen es, Prioritäten zu s.etzen
Du solltest dich auch schon in den ersten Semestern bei den Klausuren um gute Noten bemühen. Auch wenn dir gute Noten nicht besonders wichtig sind, solltest du folgendes beachten: Während man früher noch mit der Durchschnittsnote „befriedigend“ (ab 62%) einen sicheren Master-Studienplatz hatte, hat der Master nun einen NC.

Vielleicht hast du schon irgendwo anders studiert und dort Scheine erworben? Dann kannst du dir möglicherweise einige Fächer anerkennen lassen. Wenn deine Noten entsprechend gut sind, dann nimm die Chance wahr. Es könnten die Punkte sein, die dir ein Semester ersparen. 

Über  das “Microsoft Imagine” Programm (früher Dreamspark, noch früher MSDN Academic Alliance) können viele der Softwareprodukte aus dem Hause Microsoft zu Studienzwecken gratis bezogen werden.
https://msdnaa.ruhr-uni-bochum.de

Falls du den Speicherplatz deines RUB-E-Mail-Postfachs von 1GB auf 10GB erhöhen möchtest, kannst du dies eigenständig mit Hilfe des Selfcare-Interfaces tun. Dieses findest du unter: https://mail.ruhr-uni-bochum.de/mail/faq/selfcare

Erhöhte Koffeinzufuhr ist während des Studiums nicht auszuschließen. Glücklicherweise gibt es auf dem Campus unzählige Möglichkeiten, an neuen Stoff zu kommen. Zum einen gibt es in vielen Gebäuden Cafeterien (HZO, IB, ID, NA, NC, MA, Mensa-Foyer, Studierendenhaus, G-Reihe). Zum anderen bietet die Mensa (täglich geöffnet von 11:00 – 14:30 Uhr, Freitags bis 14:00) eine Vielzahl von Getränken und variierenden Speisen an. Wenn's mal etwas länger dauert, kannst du auch bis 16:00 Uhr das Bistro besuchen. 
Ein wenig edler und entsprechend kostspieliger ist ein Besuch im QWest auf der G-Seite der Universität. Hier gibt es von 9:00 bis 11:00 Uhr Frühstück, von 11:30 bis 14:30 Uhr Mittagessen, anschließend bis 18:00 Uhr Kaffee und daraufhin bis 22:00 Uhr Abendessen.
Nähere Infos zu all den genannten Einrichtungen findest du unter: http://www.akafoe.de/
Darüber hinaus gibt es noch das Uni-Center, wo du dich auch samstags oder zu späteren Uhrzeiten mit Speis und Trank
eindecken kannst.


Ungestörter Ort zum Lernen allein oder in der Gruppe gesucht? Die Uni-Bibliothek bietet hierfür einen speziellen Bereich an, in dem es nicht immer so totenstill wie im Rest der Bib sein muss.
Auch in der Mensa kann außerhalb der Betriebszeiten gepaukt werden. Von 9 bis 11 Uhr so wie von 15 bis 19:30 (Freitags bis 18 Uhr) ist der Eingang an der N-Reihe (Ostseite) geöffnet, über das Bistro oben kommt man nicht rein.
Weniger bekannt ist ein Raum im NB auf Ebene 03. Hinter einer erstmal abschreckenden Feuerschutztür findet man einen Raum mit vielen Whiteboards.