Auf den nachfolgenden Seiten erhältst du einen kleinen Überblick über die Struktur der Uni und einige wichtige Einrichtungen, die dir deine Zeit etwas erträglicher gestalten können.

---------------------------------------------------------

Die Struktur der Uni...
Im Allgemeinen gibt es an der Uni vier Interessengruppen, ohne die der Betrieb nicht möglich wäre. Die größte Gruppe sind wir, die Studenten, mit ca. 43 000 Mitgliedern. Daneben gibt es noch die Gruppe Professoren (ca. 400) und wissenschaftliche Mitarbeitern (2000) und ca. 2000 Angestellte in Technik und Verwaltung.

Organisatorisch ist die Uni recht hierarchisch aufgebaut. Die Angewandte Informatik ist einer von etwas über hundert Studiengängen an der RUB. Auf der Ebene des Studiengangs AI gibt es Gremien wie den Prüfungsausschuss und die Qualitätsverbesserungskommission, die sich um die Belange des Studiengangs kümmern. Dort haben die Studenten Mitspracherecht. Die Vertreter entstammen dabei der Fachschaft, die auf ihrer Vollversammlung einen Fachschaftsrat wählt,  der wiederum neben anderen Aufgaben, z.B. der Organisation der Erstiwoche, Mitglieder für die Gremien des Studiengangs auswählt.

... und des Studiengangs
Jeder Studiengang gehört zu einer Fakultät. Jeder Studiengang? Nein! Ein von unbeugsamen [AI]-lern bevölkerter Studiengang z.B. wird derzeit von einem Institut geleitet. In unserem Falle handelt es sich um das Institut für Neuroinformatik (INI), das seit dem Wintersemester 2013/2014 offiziell die Federführung des Studiengangs übernommen hat. Damit geht einher, dass Vieles bei der AI anders geregelt ist als bei den meisten übrigen Studiengängen. Die Leitung unseres Studiengangs obliegt jedoch wie bei allen anderen Studiengängen einem Dekan, welcher aus der Gruppe der Professoren stammt. Unser derzeitiger Dekan ist Prof. Dr. Laurenz Wiskott. Über dem Dekan stehen auf universitärer Ebene der Senat, der Hochschulrat und das Rektorat.

Auf der Ebene der studentischen Selbstverwaltung ist die zentrale Instanz das Studierendenparlament (kurz StuPa), das jährlich den AStA (Allgemeiner Studierendenausschuss) wählt. Der AStA als "Regierung" des StuPa, wiederum bildet Referate und Arbeitsgruppen für verschiedene Aufgaben.

Des Weiteren gibt es noch viele kleinere Beiräte und Kommissionen die z.B. die Einführung von neuen Studiengängen vorbereiten oder Satzung der Bibliothek ändern. Daneben gibt es dann natürlich noch den Alltagsbetrieb des Studierens und Forschens. Hier sind die wichtigsten Stationen neben der Fachschaft, das Studiendekanat und das Prüfungsamt. Verwaltet werden die Studenten im Studierendensekretariat. Außerdem gibt es Bibliotheken auf Fakultäts-, und Institutsebene aber auch die zentrale Universitätsbibliothek. Daneben existiert außerdem das Rechenzentrum, das für die elektronische Infrastruktur und, auf dezentraler Ebene, für die CIP-Inseln zuständig ist.

Nicht zu vergessen ist auch das Studierendenwerk (AKAFÖ), das sich mit den Cafeterien und der Mensa um eure Verpflegung kümmert, die Wohnheime verwaltet und eure BAföG-Anträge bearbeitet.



Hochschulpolitik
Rektorat
Das Rektorat, bestehend aus dem Rektor (Prof. Dr. Axel Schölmerich), der Kanzlerin (Dr. Christina Reinhardt) und den drei Prorektoren (Forschung, Transfer und Nachwuchs: Prof. Dr.-Ing. Andreas Ostendorf, Lehre und Internationales: Prof. Dr. Kornelia Freitag, Planung und Struktur: Prof. Dr. Uta Hohn), leitet die Universität im Alltag.

Hochschulrat
Seit dem Jahre 2008 besitzt jede Universität in NRW einen Hochschulrat. Dieser wählt die Mitglieder des Rektorats, beaufsichtigt das durch die Hochschulleitung erledigte operative Geschäft, nimmt Stellung zu Rechenschafts- und Evaluationsberichten und hat darüber hinaus beratende Funktion. Außerdem muss dem Hochschulentwicklungsplan und dem Wirtschaftsplan durch den Hochschulrat zugestimmt werden.

Senat
Der Senat setzt sich aus dem Rektor und 25 gewählten Mitgliedern zusammen, die in folgende Gruppen eingeteilt sind:

•	Professoren (13 Mitglieder)
•	Wissenschaftliche Mitarbeiter(4 Mitglieder)
•	Mitarbeiter in Technik + Verwaltung (4 Mitglieder)
•	Studenten (4 Mitglieder)

Der Senat hat ein weitgestreutes Aufgabenfeld. Zum einen bestätigt er die Wahl der Mitglieder des Rektorats, zum anderen kann er diverse Ordnungen erlassen und ändern sowie über eine Menge andere Dinge Empfehlungen aussprechen und Stellungnahmen geben.

Die Mehrheitsverhältnisse sind leider recht unfair verteilt. Trotzdem solltet ihr euer Wahlrecht wahrnehmen und jährlich (meist im Juni) die studentischen Mitglieder mitwählen. Der Senat tagt monatlich öffentlich im Senatssitzungssaal in der Universitätsverwaltung (UV).


Studierendensekretariat
Das Studierendensekretariat verwaltet alle Studenten an der Universität. Falls du vor hast deinen Studiengang zu wechseln, dich vom Bachelor in den Master umzuschreiben oder du einfach nur eine Studienbescheinigungfiref benötigst, ist das Studierendensekretariat die richtige Adresse. Du findest es im im Gebäude SSC (Studierenden-Service-Center) 0/229. Öffnungszeiten: Mo – Fr: 9 – 12 Uhr; Mo, Mi, Do: 13:30 – 15 Uhr

Dekan
Unser aktueller Dekan ist Herr Prof. Dr. Laurenz Wiskott. Der Dekan ist das Oberhaupt des Studienganges. Der Fachschaftsrat steht in engem Kontakt zum Dekan und vertritt ihm gegenüber die Interessen der Studenten.

Studiendekanat
Die Leiterin des Studiendekanates, Frau Kallweit, sitzt im NB-Gebäude im Raum 02/72. Sarah Thiele ist studentische Mitarbeiterin im Dekanat und leistet die studentische Studienberatung. Zum Studiendekanat geht man, wenn man die Studienberatung in Anspruch nehmen möchte oder allgemeinen Fragen zum Studiengang oder der Prüfungsordnung hat. Offene Fragen bezüglich des Studienganges können in der Regel schnell beantwortet werden. 
Sprechzeiten: Di 12-14 Uhr, Do: 9 – 11 Uhr.

Fakultätsrat
Der Fakultätsrat entscheidet über die Details der Studiengänge, schlägt neue ProfessorInnen vor und entscheidet in letzter Instanz über die Belange der Studiengänge, wie die Prüfungsordnungen oder auch die Verwendung der Studiengebühren. Die Mehrheitsverhältnisse sind ähnlich denen im Senat.

Prüfungsausschuss (PA)
Im Prüfungsausschuss sitzen ebenfalls Professorinnen und Professoren, wissenschaftliche Mitarbeiterinnen und Mitarbeiter sowie Studenten. Im Prüfungsausschuss wird neben dem 
Hauptthema „Organisation von Prüfungen“, auch über die Anerkennung von Prüfungsleistungen, sowie über den möglichen Austausch von Studienfächern gesprochen. An den Prüfungsausschuss 
richtet ihr alle Anträge z.B. zur Anerkennung von Prüfungsleistungen, Härtefallanträge bei drei nicht bestandenen Versuchen in einer Prüfung, nach einem Fachwechsel oder beim Einstieg in den Master. Der Vorsitzende des Prüfungsausschusses ist derzeit Prof. Dr. Markus König von den Bauingenieuren.

Prüfungsamt
Das Prüfungsamt ist die erste Anlaufstelle für Prüfungsangelegenheiten. Bei Frau Nawrat und Frau Füllbeck könnt ihr euch für Prüfungen anmelden (innerhalb der so genanten Anmeldefrist) oder abmelden (mindestens zwei Wochen vor der Prüfung, besser früher!) und Auszüge mit der Übersicht über die erreichten Studienleistungen und Scheine erhalten. Inzwischen können die Klausuranmeldungen auch online über das „FlexNow“-Portal vorgenommen werden. Das Prüfungsamt befindet sich im GA 03/41.
Sprechzeiten: Mo: 10 – 12 Uhr, Mi: 11-13 Uhr

Qualitätsverbesserungskommission (QVK)
Die Zuteilung der Qualitätsverbesserungsmittel an die verschiedenen Fakultäten und Dozenten obliegt der Qualitätsverbesserungskommission. Auch hier haben studentische Vertreter ein Mitspracherecht, welches durch vom FSR entsandte Vertreter(innen) in Anspruch genommen wird.

Die Fachschaft (FS)
Die Fachschaft sind alle Studenten im entsprechenden Studienfach. Also auch du! Meist bilden sich deine Lern- und Projektgruppen aus der Fachschaft. Durch gemeinsame Veranstaltungen wie die Kneipentour, die Fachschaftsfahrt und die Weihnachtsfeier wird das Zusammengehörigkeitsgefühl der AI-ler gestärkt.

Die Vollversammlung (VV)
Auch wenn ihr nicht vorhabt, den Studiengang aktiv mit zu gestalten, sondern „einfach nur studieren“ wollt, solltet ihr trotzdem diese eine Veranstaltung auf jeden Fall besuchen: Die VV der Fachschaft, die am Anfang eines jeden Semesters stattfindet. Hier werden wichtige Informationen über den Studiengang mitgeteilt und wichtige Entscheidungen gefällt. Darüber hinaus wählt die Vollversammlung den Fachschaftsrat und beauftragt ihn mit Arbeitsaufträgen, die er im Laufe des anschließenden Semesters zu erfüllen hat. Es ist also die ideale Gelegenheit, Verbesserungsvorschläge für den Studiengang einzubringen und zu lösende Probleme anzusprechen. Gerüchteweise soll es  auf der VV als weiteren Anreiz auch Kekse und Bier geben!

Der Fachschaftsrat (FSR)
Der FSR ist ein auf der VV gewähltes Gremium und untersteht der Fachschaft. Die Mitglieder wollen das umsetzen, was die VV ihnen an Arbeitsaufträgen auferlegt hat. Der FSR vertritt die Fachschaft gegenüber Gremien wie dem Prüfungsausschuss oder dem Dekan und setzt sich für die Interessen der Studenten ein. Solltest du Fragen bezüglich deines Studiums haben, kann dir der FSR helfen oder dich an eine fachkundige Person weitervermitteln. Jeder Student der Angewandten Informatik kann an der Fachschaftsratssitzung teilnehmen und ist stimmberechtigt.

FachschaftsvertreterInnenkonferenz (FSVK)
Die FachschaftsvertreterInnenkonferenz ist ein Gremium dass niemals außen tagt. Hier treffen sich Vertreterinnen und Vertreter der einzelnen Fachschaften (bzw. meist aus den Fachschaftsräten), um sich gegenseitig über die aktuelle Lage ihres Studiengangs zu informieren, um die gemeinsame Arbeit zu koordinieren oder auch um der studentischen Senatsfraktion ihr Votum mitzuteilen.

Studierendenparlament
Das Studierendenparlament (StuPa) ist das höchte Gremium in der studentischen Selbstverwaltung. Hier wird einmal jährlich der AStA gewählt, der Haushalt geprüft oder auch Entscheidungen zum Semesterticket gefällt. Das Studierendenparlament besteht aus 35 Mitgliedern die verschiedenen Listen angehören. Die Wahlen zum Studierendenparlament finden jährlich am Ende des Wintersemesters statt. Da die Wahlbeteiligung bisher meist sehr gering war, seid ihr aufgefordert das zu ändern. Die Stimme des Studierendenparlaments hat nämlich nur dann ein hohes Gewicht, wenn es von ausreichend Studenten legitimiert ist.

AStA
Der Allgemeine Studierendenausschuss wird vom StuPa gewählt. Der AStA verfügt über das Geld der Studentenschaft. Momentan geht ein Teil eures Semesterbeitrags an den AStA, der damit verschiedene Veranstaltungen finanziert, aber auch eine Rechts-, AusländerInnen- und BaföG- Beratung anbietet. Außerdem unterhält der AStA das Kulturcafe, in dem häufig Veranstaltungen und Parties stattfinden sowie zwei Copyshops (in GA 03 und GB 02). Der AStA vertritt die Studentenschaft gegenüber der Öffentlichkeit. Im AStA-Flur im Studierendenhaus sind die verschiedenen Referate angesiedelt, in denen ihr auch Informationen und Beratung zu wichtigen Dingen des Studi-Alltags erhaltet (Finanzen, Wohnungssuche usw.). Übrigens stellt der AStA auch den Internationalen Studierendenausweis aus, der euch in vielen Ländern weltweit Vergünstigungen bringt.

AKAFÖ
Das akademische Förderungswerk kümmert sich um die wichtigen Details des Studentenlebens. Es betreibt die Mensen und Cafeterien auf dem Campus. Daneben ist es noch für die Wohnheime und das Bafög zuständig. Die Verwaltung des AKAFÖs und das BAföG-Amt findest du im Studierendenhaus. Boskop wird übrigens auch vom AKAFÖ finanziert. Das Akafö finanziert sich zum Teil durch den Sozialbeitrag. 105 € davon fließen dorthin.

CIP-Insel
In den mit CIP-Pool gekennzeichneten Räumen im Gebäude ID stehen Studenten der Angewandten Informatik Rechner zur freien Verfügung. Den entsprechenden Account bekommt man unter Vorlage einer gültigen Studienbescheinigung vor Ort. Die Tutoren werden in der ersten Woche die Anmeldung mit euch vornehmen. Die CIP-Insel hat meist von 10-18 Uhr geöffnet.

Rechenzentrum
Das Rechenzentrum stellt das informationstechnische Herz der Uni dar. Interessante Aspekte sind vor allem der Internetzugang auf dem Campus (per WLAN oder HIRN Port), der Download von campuslizensierter Software (z.B. Sophos Antivirus, Windows) und der Erwerb bzw. das Leasing von Laptops zu rabattierten Preisen.

Ausführliche Informationen findet ihr auf der Homepage: https://www.rz.ruhr-uni-bochum.de

Bibliothek
Die Bibliothek ist sehr zentral auf dem Campus angesiedelt (wenn ihr von der Uni-Brücke geradeaus in Richtung Uni lauft, landet ihr quasi direkt vor dem Gebäude). 

In der Bibliothek findet ihr jede Menge Bücher, Zeitschriften, Dissertationen, etc. Um etwas auszuleihen, braucht ihr lediglich euren Studierendenausweis. 

Da ihr die Bibliothek weder mit Taschen noch mit Jacken betreten dürft, empfiehlt sich die Mitnahme einer 2-Euro-Münze zwecks Anmietung eines Spindfachs, denn die Bibliothek wechselt nicht.

Im Foyer befindet sich außerdem noch ein Café, eine Etage tiefer Toiletten und innerhalb der Bibliothek einige Computerarbeitsplätze zur Buchrecherche.

Jeden Mittwoch um 12:15 Uhr gibt es eine Bibliotheksführung. 

Öffnungszeiten: Mo – Fr: 8:00-24:00 Uhr, Sa: 11:00-20:00 Uhr & So: 11:00-18:00 Uhr (ab 22 Uhr und Sonntags ist jedoch kein Servicepersonal anwesend)

Für weitere Infos siehe Kapitel „Links“.