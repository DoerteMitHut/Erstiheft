BAföG
„BAföG“ steht für Bundes-Ausbildungsförderungs-Gesetz. Dahinter verbirgt sich unter anderem eine Möglichkeit zur Studienfinanzierung für Studenten mit geringem Einkommen und Vermögen. Die gesetzlichen Hintergründe und Vorschriften sind zu komplex, um sie im Rahmen dieses Heftes wiederzugeben, zumal für nahezu jeden Studierenden irgendwelche Ausnahmen und Sonderregelungen greifen. Deshalb nur die beiden wichtigsten Aussagen:

BAföG wird nicht rückwirkend gezahlt! Bzw. nur rückwirkend bis zu dem Monat in dem du den Antrag eingereicht hast.

Da wird dir geholfen: Wer keinen hilfsbereiten BAföG Berater beim Akafö erwischt und Hilfe braucht, sollte die BAföG-Beratung des AStA aufsuchen und sich dort kompetent beraten lassen!
Oder googlen :)

Stipendien
Viele Organisationen haben es sich zur Aufgabe gemacht, Studierenden mit Stipendien zu fördern. Da dies nur wenige Studierende in Anspruch nehmen, lohnt es sich auf jeden Fall, ein Stipendium zu beantragen.

Grundsätzlich fördern die meisten Stiftungen analog dem BAföG-Satz (aber man muss eben im Gegensatz zum BAföG später nichts zurückzahlen). Darüber hinaus gibt es i.A. eine „ideelle“ Förderung in Form von Büchergeldern und Angeboten zur Teilnahme an besonderen Veranstaltungen. Gerade bei den Veranstaltungen wird dann aber auch erwartet, dass man regelmäßig teilnimmt. Normalerweise sind auch regelmäßige Berichte anzufertigen, in denen man seinen Studienfortgang kommentiert.

Studierende aus dem Ausland
Der DAAD fördert Studierende aus allen Ländern der Welt bei Aus- und Fortbildung sowie Forschungsarbeiten in allen Fachrichtungen. Eignungsvoraussetzung: Abgelegte Zwischenprüfung oder Vordiplom, Deutschkenntnisse. Bewerbung i.d.R. nur im Heimatland beim zuständigen Kultus-/Bildungs- oder Hochschulministerium, in Deutschland beim Akademischen Auslandsamt der zuletzt besuchten Hochschulen (wenn Vordiplom schon in Deutschland gemacht wurde).

Die parteinahen Stiftungen fördern ebenfalls zum Teil Ausländerinnen und Ausländer.

Parteinahe Stiftungen
Alle im Bundestag vertretenen Parteien haben parteinahe Stiftungen gegründet, die auch besonders begabte Studierende, die sich gesellschaftlich engagieren, fördern.

Je nach nahe stehender Partei der Stiftung wird dabei auf unterschiedliche Dinge Wert gelegt. Hier kann euch oft die Hochschule weiterhelfen.

Konfessionelle Träger
Die Förderung der kirchlichen Studienwerke ist an den entsprechenden Glauben gebunden. Auch hier kann die Förderung erst im Studium einsetzen, mit der Bewerbung müssen Gutachten der Hochschule vorgelegt werden.

Wirtschaftsnahe Organisationen
Auch diverse Unternehmen und Wirtschaftsverbände haben Stiftungen oder ähnliches gegründet, die unter bestimmten Umständen auch Studienförderung leisten.

Stipendienprogramm der RUB
Inzwischen hat die Ruhr-Universität ein eigenes Stipendienprogramm, das aktuell 177 Stipendien vergibt. Für unsere Fakultät zählen gute Noten und soziales Engagement als entscheidende Faktoren. Ihr benötigt keine Empfehlung eines Dozenten oder Professors.

Das Stipendium der RUB ist als eines der wenigen Stipendien unabhängig vom eigenen Einkommen oder dem Einkommen der Eltern. Sofern man das Stipendium bekommt, erhält man 300 € pro Monat über einen Zeitraum von einem Jahr.

http://www.ruhr-uni-bochum.de/bildungsfonds/

Sozialbeitrag / Semesterbeitrag
Nicht zu verwechseln mit den (abgeschafften) Studiengebühren, auch wenn es Ähnlichkeiten gibt. Der Sozialbeitrag muss jedes Semester entrichtet werden und bewegt sich zur Zeit in der Größenordnung von 300€. Davon entfallen 182€ auf das Semesterticket, 105€ gehen für Mensa, Wohnheime & Co an das AkaFö und 16€ an den AStA.
Man sollte nicht vergessen ihn rechtzeitig zu überweisen, denn die Mahnung dazu kommt meist in Begleitung einer (vorläufigen!) Exmatrikulations-bescheinigung. Wer sich das ersparen möchte, kann am Lastschriftverfahren teilnehmen, bei dem immer ca. einen Monat vor Semesterbeginn automatisch abgebucht wird.

Krankenversicherung
Jeder Student muss krankenversichert sein, was bei der Einschreibung ja auch kontrolliert wird. Die meisten Studenten sind am Anfang noch über ihre Eltern in einer sog. gesetzlichen Familienversicherung versichert. Aufpassen sollte man jedoch, wenn man bereits berufstätig ist, denn nur bis max. 425€ bzw. 450€ (bei Minijob) pro Monat bleibt dieser Versicherungsschutz erhalten, und auch dann nur bis zu einem Alter von max. 25 Jahren. Am besten mit der eigenen Krankenkasse abklären.

Darüber hinaus kann man als Student auch eine eigene Versicherung zu vergünstigten Konditionen abschließen. Die Höhe der Beitragssätze sind bei den gesetzlichen Versicherung auf ungf.  85€ pro Monat (inkl. Pflegeversicherung) und bei den privaten Versicherern auf ungf. 60-86€ festgelegt. Es existiert also ein Verdienstbereich, in dem ein Plus an Einkommen ein faktisches Minus bedeutet, weil die zu zahlenden 85€ nicht aufgewogen werden.

Wir können nur raten sich hier intensiv schlau zu machen, denn Krankenkassen können sich auch rückwirkend Leistungen rückzahlen lassen!

Mehr Infos siehe Kapitel „Links“