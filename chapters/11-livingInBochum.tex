Wohnheime
Obwohl als Pendler-Uni bekannt, gibt es rund um die RUB ein vielfältiges Angebot an Studierendenwohnheimen. Egal ob ein Zimmer in einer Wohngemeinschaft (WG), ein eigenes Appartement oder ein Einzelzimmer auf einer Gemeinschaftsetage - Studierende der Bochumer Hochschulen sowie der FH Gelsenkirchen können aus einem großen Angebot an hochschulnahem und preisgünstigem Wohnraum auswählen.

Das AKAFÖ bietet sowohl Zimmer als auch Appartements in 18 Wohnheimen an. Alle liegen in der Nähe der Ruhr-Uni oder den anderen Hochschulen in Bochum. Die Kosten betragen „warm“ zwischen 150 Euro und bis zu 490 Euro für eine 3-Raum Wohnung. Zusätzlicher Anreiz ist der Anschluss an das Wohnheimnetz und die Hochgeschwindkeitsverbindung ins Internet. Fairerweise muss man hier allerdings sagen, dass man keine echte „Flatrate“ bekommt.

Das AKAFÖ vergibt auch Einzelzimmer in Großwohngemeinschaften. Hier hat man die Wahl zwischen Zimmer von 12-16m² Größe, die mit einem Waschbecken ausgestattet sind. Bad und Küche teilt man sich allerdings mit 8-12 Leuten von der selben Etage.

Darüberhinaus gibt es aber auch Zimmer in 2-er, 3-er oder 4-er WGs (zB. Die Wohnheime «Studidorf Laerheide» oder «Europahaus»), in denen meistens die Sympathie entscheidet, ob man das Zimmer letztendlich bekommt oder nicht.

Wichtigste Voraussetzung, um ein Zimmer in den vom AKAFÖ verwalteten Gebäuden zu bekommen: Es muss rechtzeitig ein Online-Antrag gestellt werden. Danach heißt es: Geduld haben.

Insider-Tipp: Die netten Sachbearbeiter beim AKAFÖ (zu finden im Studierendenhaus) gelegentlich telefonisch oder mit einem Besuch daran erinnern, dass man auf der Suche ist! Dann kann es sein, dass dein Antrag etwas schneller bearbeitet wird.

Private Wohnheime
Neben den staatlich geförderten AKAFÖ Wohnheimen, gibt es auch einige private Wohnheime, die z.B. von verschiedenen Vereinen, Wohnungsbaugesellschaften oder anderen Förderungswerken verwaltet werden. Hier kann man Zimmer zwischen 150 und 270 Euro mieten, allerdings muss man sich für jedes Wohnheim einzeln bewerben.

Vorsicht ist geboten bei Angeboten von sog. Verbindungen. Hier kann man zwar oft günstig wohnen, muss dafür aber einer solchen Verbindung (oft lebenslang) beitreten und an deren Veranstaltungen teilnehmen, die manchmal recht konservativ erscheinen.

Selber Suchen
Alle die lieber alleine wohnen, mit anderen Leuten eine WG gründen oder in eine bestehende einziehen, finden immer einen Haufen Wohnungsanzeigen, entweder direkt an den schwarzen Brettern in der Uni, im Internet oder z.B im Stadtspiegel.

Da in Bochum, wie auch in den meisten anderen Städten im Ruhrgebiet, kein Wohnraummangel herrscht, gibt es eine Menge bezahlbarer Wohnungen. Bei der Suche sollte man die zusätzlichen Kosten für Telefon und Internet, sowie Heizung, Strom, Wasser und eventuell Gas im Hinterkopf behalten (Nebenkosten schimpft sich das).

Ein-Personen-Wohnungen gibt es außerhalb der Innenstadt oft ab ca. 300 Euro. Wer in eine WG zieht, kann auch zu Preisen wohnen, die ähnlich denen in Wohnheimen sind. Die meisten Inserate findet man übrigens im Internet.

Mehr Infos siehe Kapitel „Links“