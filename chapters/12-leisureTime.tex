Sport
Für jeden, auf den das Motto „Sport ist Mord“ nicht zutrifft, hat das Angebot des RUB Hochschulsports etwas parat. Die RUB verfügt über mehrere große Sporthallen, welche sich unterhalb der Mensa befinden, sowie Außensportanlagen an der Markstraße. Die Plätze stehen zu bestimmten Zeiten frei zu Verfügung. Dort kann man mit KommilitonInnen oder Studierenden anderer Richtungen gespielt werden.

Neben diesen Sportarten sind im Sportangebot des Hochschulsports auch Kurse mit Trainern im Angebot. Hier kann unter anderem Fechten, Karate und Trampolin springen erlernt werden.

Sehr beliebt sind auch die allgemeinen Fitnesskurse, welche mit Laufen, Krafttraining und Dehnübungen dem Körper Kraft und Ausdauer verleihen sollen. Auch im Wassersport-Bereich ist das Angebot groß. Die RUB verfügt über ein Hallenbad im Uni Center. Dort werden mehrere Schwimmkurse angeboten. Um daran teilzunehmen, muss man am Anfang des Semesters bei der Einteilung in die Schwimmgruppen dabei sein. Im Hallenbad können auch andere Sportarten wie das Unterwasser- Rugby und Tauchen betrieben werden.

Alle angebotenen Kurse sowie Trainingszeiten findet ihr auf der Homepage des Hochschulsports. 

Kultur an der RUB
Boskop (manchmal „boSKop“ geschrieben) ist die „bochumer Studentische Kulturoperative“, des Kulturbüros vom AKAFÖ und damit beauftragt, an den Bochumer Hochschulen studentische Kultur anzuregen und zu fördern. Dazu bietet Boskop eine Vielzahl von interessanten Workshops, internationale Kulturtreffen, musikalische Aufführungen und Themenabende an.

Im Kultur Café direkt auf dem Campus Gelände wird monatlich die Blues Session Bochum angeboten. Dort treten wechselnde Jazz und Blues Bands auf und im Anschluss findet meist ein freies „Jammen“ statt.

Wer Interesse an Internationalen Filmen des Ostens hat, kann den wöchentlich stattfindenden osteuropäischen Film-abend kostenlos besuchen.

Besonders interessant sind die Workshops: Sie laufen in der Regel ein Semester lang. Hier kann man z.B. die Kunst des Cocktailmixens erlernen, sich mit anderen über Literatur unterhalten oder sich im kreativen Schreiben üben. Hier werden auch Sport und Tanzarten aus fremden Ländern wie Capoeira, Tango und orientalischer Tanz gelehrt. Die Anmeldung für die Workshops findet i.d.R. am Anfang des Semesters im Foyer der Mensa statt.

Wer erstmal mit den Standardtänzen anfangen möchte, dem seien die Tanzkurse des AStA ans Herz gelegt.

Daneben gibt es noch viele andere kulturelle Initiativen. Z.B. findet im Sommer das internationale Videofestival statt. Der Studienkreis Film (SKF) bestimmt das wöchentliche Kinoprogramm, welches im HZO 20 gezeigt wird. Der wirklich kostengünstige Besuch im Unikino ist auf jeden Fall lohnenswert. Es ist zu empfehlen, sich ein Kissen mitzubringen!


Kneipen
Das Bermuda-3-Eck!
Das „Bermuda-3-Eck“, wie vor allem die Ecke der Innenstadt rund um den Engelbertbrunnen genannt wird, erfreut sich großer Beliebtheit, und das nicht nur am Abend. Um ein paar der vielen verschiedenen Kneipen kennen zu lernen, empfehlen wir unsere Kneipen-Tour in deiner ersten Uniwoche.

Absinth
Rottstr. 24, 44793 Bochum (Nähe Rotlichtviertel). Urige Kneipe mit buntgemischtem Publikum. Und, wie der Name schon verrät, großer Absinth-Auswahl!

Kultur Café
Größter Vorteil: direkt an der Uni. Perfekt geeignet zum Lernen, gemütlich einen Kaffee trinken, Leute treffen, sowie ein Bier vor, zwischen oder nach den Vorlesungen. Abends gibt es dort auch kulturelle oder politische Veranstaltungen.


Wohnheimkneipen
Hierbei handelt es sich um Kneipen in Wohnheimen für Studierende. Diese werden i.d.R. von den Bewohnern geführt und glänzen nicht nur durch Gemütlichkeit, sondern auch durch gute Preise. Leider öffnen und schließen jedes Jahr ein paar Wohnheimkneipen, so dass wir einfach keinen Überblick mehr darüber haben, welche gerade noch existiert und welche nicht. Fragt einfach rum und haltet die Ohren auf.

Tipp: Häufig haben diese Kneipen nur an bestimmten Wochentagen geöffnet.


Discotheken & Clubs
So, und wenn euch das jetzt immer noch nicht genug ist, hier noch ein paar Tipps zur Wochenend- und Freizeitgestaltung in Bochum:

Matrix Rockpalast (Hauptstr. 200, 44892  Bochum): Gothic bis Punk
Hier werden sehr viele Musikwünsche befriedigt und je nachdem was gerade für ein Special ist, kommt man auch umsonst rein. Dazu werden hier teilweise Konzerte gespielt.
http://www.matrix-bochum.de

Untergrund (Kortumstr. 101, 44787 Bochum): Samstags Alternative, Rock und Indie, Freitags Events und Gemischtwaren
Der Untergrund befindet sich quasi mitten in der Stadt Bochum und der Eingang ist teilweise zu übersehen. Wenn man jedoch erst einmal drin ist und die Stufen nach unten gemeistert hat, erwartet einen eine kleine Tanzfläche. Für nähere Infos hängen auch an der Uni sehr oft Plakate aus, auf denen dann auch die jeweils gespielte Richtung angegeben wird.
http://www.myspace.com/untergrundclub


Schwimmbäder
Aquaris Schwimmbad und Saunaworld (Herner Straße 299, 44809 Bochum):
http://www.aquaris.de

Hallenbad Querenburg „Uni-Bad“ (Hustadtring. 157, 44801 Bochum)
Morgens stark ermäßigt für Studenten

Freizeitbad Heveney (Kemnader See, Querenburger Strasse 35, 58455 Witten): 
http://www.kemnadersee.de



Kinos
Bofimax-Kinocenter (Kortumstr. 51, 44787 Bochum):
http://bofimax.de/

Casablanca Filmtheater (Kortumstr. 11, 44787 Bochum):
http://www.casablanca-bochum.de/

Union Kino (Kortumstr. 16, 44787 Bochum)
http://kino-bochum.de/

UCI Kinowelt (Ruhr Park)
http://www.uci-kinowelt.de/

Studienkreis Film („SKF”, RUB):  Von Studenten für Studenten
http://dbs-lin.ruhr-uni-bochum.de/skf/

Theater
ET CETERA Variete (Herner Str. 299, 44809 Bochum):
http://www.variete-et-cetera.de

Prinz-Regent-Theater (Prinz-Regent-Str. 50 – 60, 44795 Bochum):
http://www.prinzregenttheater.de

Schauspielhaus Bochum (Königsallee 15, 44789 Bochum):
http://www.schauspielhausbochum.de