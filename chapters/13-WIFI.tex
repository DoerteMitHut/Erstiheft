Es soll ja Leute geben, die einen Laptop in die Uni schleppen. Wir alle. 

Da ein Computer ohne Internet doch recht langweilig ist, hier ein kurze Zusammenfassung, damit du ins Netz kommst. Dafür hast du 3 Möglichkeiten:

Per HIRN-Port
Mit einem normalen Netzwerkkabel. Einfach in eine mit H.I.R.N. gekennzeichnete Dose einstöpseln und sich automatische eine IP geben lassen.

Wenn du das erste Mal eine Internetseite aufrufst, wirst du auf eine Loginseite umgeleitet. Hier gibst du deine LoginID und dein Passwort ein. Sollte das nicht klappen, kannst du auch login.rz.rub.de manuell aufrufen. Los geht‘s…

Per WLAN und VPN
Du verbindest dich mit dem Accesspoint RUB-WLAN. Natürlich musst du dir deine IP wieder automatisch zuweisen lassen.

Sobald du surfen willst, wirst du auf eine spezielle Seite umgeleitet, denn ohne die spezielle Cisco Zusatzsoftware (VPN) kannst du nur auf manchen internen Seiten surfen. Die Startseite beschreibt für die nötigen Schritte.

Vorteil hier: Du kannst dich mit derselben Software auch von Zuhause in das Uni-Netz einklinken, um so z.B. an bestimmte Dokumente zu gelangen.

Per WLAN mit eduroam
Dies ist die zu bevorzugende Methode. Hierfür benötigst du keine Zusatzsoftware. Außerdem solltest du damit auch an anderen Unis surfen können.

Alle nötigen Schritte findest du hier:
https://noc.rub.de/web/wlan

Hinweis:
Achte auch deine freigegeben Ordner, es soll nämlich tatsächlich Leute auf dem Campus geben, die einfach mal das Netzwerk nach allen Freigaben durchsuchen.

Übersichtsseite über die WLAN Möglichkeiten des Rechenzentrums:

http://www.rz.ruhr-uni-bochum.de/dienste/netze/wlan/
