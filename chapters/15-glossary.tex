0
Vorangestellt kennzeichnet die „0“ in den Gebäuden die Etagen unterhalb der Forumsebene. Ein Erdgeschoss, also eine 0. Etage selber, gibt es (außer im UFO) jedoch nicht, die Zählung beginnt oberhalb der Gebäudemitte sofort bei 1.

42
Die Antwort auf die universelle Frage nach dem Leben, dem Universum und allem. Genügt nicht zum Bestehen.

AI
•	Angewandte Informatik — dein Studiengang.
•	Anonyme Informatiker — Selbsthilfegruppe
•	Amnesty International — helfen uns nicht, trotz der vorherrschenden Zustände
•	Artificial Intelligence — Künstliche Intelligenz

Allgemeiner Studierenden-Ausschuss (AStA)
Studentische Interessensvertretung auf Uni-Ebene. Wird vom Studierendenparlament einmal im Jahr gewählt. Den AStA findest du im Studierendenhaus gegenüber der Uni-Verwaltung.

Akademisches Förderungswerk (AKAFÖ)
Verantwortlich für Mensen, Cafeten, staatliche Wohnheime und andere Dinge, die das Studi-Herz begeistern. Zu finden im Studierendenhaus in den Räumen 059, 060 und 056.

Audimax
Auditorium Maximum (lat. “Das größte Publikum”). Der größte Hörsaal der Uni. An der RUB das runde Gebäude in der Mitte, soll eine Muschel darstellen (kein Scherz).

Beurlaubung
Aus welchen Gründen auch immer du dich für ein Semester beurlauben lassen willst, diese Beurlaubung musst du im Uni-Sekretariat beantragen und genehmigen lassen. Die Urlaubssemester werden nicht auf die Studiendauer angerechnet und der Studienplatz bleibt erhalten.

Bochumer Studentische Kulturoperative (Boskop / BoSKop)
Unterstützer und Veranstalter vieler studentischen Kulturaktivitäten (Video, Literatur, Malen, Gestalten, 39 Theater u.s.w.). Sitzt im Wohnheim Sumperkamp 9-15. Anmeldungen für Kurse finden auch dort statt.

Botanischer Garten
Grünanlage im Süden des Unigeländes. Die Frage, ob es auch nicht-botanische Gärten gibt, konnte noch nicht abschließend geklärt werden.

Bundes-Ausbildungsförderungsgesetz (BAföG)
Gesetz, welches regelt, dass du keine oder nur unzureichende Ausbildungsförderung erhältst. Das für dich zuständige BAföG-Amt befindet sich im Uni-Verwaltungsgebäude auf der Eingangsebene. Bei Fragen oder Schwierigkeiten wende dich bitte an die BAföG-Beratung des AStA.

CCC - Chaos Computer Club
1981 gegründeter deutscher Verein, in dem sich Hacker zusammengeschlossen haben und inzwischen ca. 2000 Mitgliedern hat.

Caféte
Länger als die Mensa geöffnete Anlaufstellen für den kleinen Hunger oder Durst nebenbei. Caféten befinden sich verteilt auf dem ganzen Campus.

CIP-Insel
„Computer Investment Program“. Eine Ansammlung von Computern, auch auf dem ganzen Campus.

Credit Point (CP)
Bewertungskriterium für Studienleistungen, ein CP entspricht etwa 30 Arbeitsstunden. 180 braucht man für den Bachelor, 120 für den Master.

Dekan, Dekanat
Der Dekan führt die Geschäfte eines Studienganges und vertritt ihn innerhalb der Hochschule, gegenwärtig Prof. Dr. Laurenz Wiskott.

Deutsches Forschungsnetz (DFN)
Schnelles Backbone-Netz, das unter anderem die Unis verbindet.

Ersti, S. 1 ff.
Wenn du das hier liest, bist du höchstwahrscheinlich einer.

F___
Kennzeichnet in Raumnummern die Flachbauten zwischen den eigentlichen Gebäuden. Der zweite Buchstabe gibt an, ob der Flachbau westlich (W) oder östlich (O) des Gebäudes liegt. ICFW beispielsweise ist der Flachbau, der sich vom Forum aus vor dem Gebäude IC befindet.

Fachschaft (FS)
Zusammenschluss aller Studenten eines Studienganges, in diesem Fall also alle Studenten des Studienganges Angewandte Informatik.

Fachschaftsrat (FSR)
Der auf der Vollversammlung gewählte Fachschaftsrat setzt eure Interessen gegenüber der Uni-Verwaltung und dem AStA durch. Sollte für dich erster Ansprechpartner bei Fragen oder Problemen aller Art sein.

Fachschafts-VertreterInnen-Konferenz (FSVK)
Regelmäßig zusammentretendes Gremium aus Vertretern aller Fachschaften. Koordiniert die Fachschaftsarbeit und entschiedet über Anträge.

Fakultätsrat (Fakrat)
Wird einmal im Jahr (meist im Juni) bei den Gremienwahlen gewählt. Er setzt sich aus acht Professoren, zwei wissenschaftlichen Mitarbeitern, zwei nicht-wissenschaftlichen Mitarbeitern und drei Studenten zusammen. Vorsitzender ist der Dekan. Der Fakultätsrat ist das oberste beschlussfähige Gremium einer Fakultät. Hier finden Verhandlungen über Studienordnungen, Lehrpläne und Berufungen von Professoren statt. Bei uns übernimmt diese Aufgaben der Gemeinsam beschließende Ausschuss.

FlexNow
Online-Tool, mit dem Studenten ihre Prüfungen selbstständig An- und Abmelden können (in der Theorie). Funktioniert nicht mit allen Prüfungen, mit allen Browsern oder bei Vollmond. Kann mit den Rechnern der CIP-Insel genutzt werden.

Forum
•	Die universelle Plattform für den Austausch zwischen allen Studierenden, live von unserem Fachschafts-Server. Hier findet Ihr die brandheißen Informationen als erstes.
https://fs.ai.rub.de/forum
•	Bezeichnung für die Mitte der Uni (also der Platz zwischen Audimax, Universitäts-Bibliothek und dem HZO), der bei der Definition des Forumslevel eine gewaltige Rolle spielt.

Forumslevel
•	Normalnull der Uni, alle Etagenbezeichnungen in den Gebäuden werden relativ zum Forum gerechnet.
•	Noch zu erstellende Map für diverse Spiele, um Uni-Frust abzureagieren.

Fundbüro
Das Fundbüro der Uni ist gleichzeitig der Infopoint im Computerpool im Eingangsbereich der Universitätsverwaltung.

Gemeinsam beschließender Ausschuss (GBA)
Der Gemeinsame beschließende Ausschuss entspricht in der AI dem ->Fakultätsrat

HIRN, HIRN-Port
HochschulInternes RechnerNetz, kommt jeder Student rein, entweder über eine LAN-Dose (HIRN-Port) oder WLan (per eduroam oder RUB-WLAN), wenn man denn Empfang hat. Oh RZ, lass HIRN regnen!

Hochschulrat
Seit 2008 höchstes Gremium der Uni.

IC
Gebäude, wurde wegen PCB-Belastung kernsaniert. Danach konnte man nicht rein wegen PCB-Belastung.

International Office (IO)
Das International Office koordiniert die internationalen Beziehungen der Universität und ist Ansprechpartner für alle Fragen rund um die Internationalität von Lehre und Forschung. 

Institut
Eine kleine Selbstverwaltungseinheit in den Abteilungen / Fakultäten. Gliedert sich meist nach wissenschaftlichen Tätigkeiten.

Java
•	Amerikanischer Slangausdruck für Straßencafe.
•	Nach 1.) benannte Programmiersprache von SUN (mittlerweile Oracle).
•   Inselgruppe, die diesen Namen wohl bald abgeben muss, weil sie die Lizenzgebühren an Oracle nicht mehr zahlen kann.

Kanzler
Der oberste Verwaltungsbeamte der Ruhr-Universität.

Konferenz der Informatik-Fachschaften (KIF)
Informations- und Aktionsplattform für Vertreter aller deutschsprachigen Informatik-Fachschaften. Quelle für lustige Plüschtier-Nähanleitungen.

Matrikelnummer
Ist auf dem Studierendenausweis aufgedruckt und wird beim Ausfüllen vieler Formulare, sowie bei den Klausuren benötigt. Kannst du bald auswendig.

Mensa
Nahrungsaufnahmestätte hinter dem Audimax mit täglich wechselnden Gerichten, Nudeln gibt es immer (an der Nudeltheke). Man kann wählen zwischen zwei Sprintern (Salat im Preis enthalten), zwei Komponentenessen (Beilage gegen Aufpreis) und dem Aktionsmenü (teuer).

N.N.
Abk. (nomen nominandum) wird immer dann verwendet, wenn die ausführende Person noch nicht feststeht.

Prüfungsamt
Verwaltet unsere Prüfungsergebnisse und ist erste Anlaufstelle für Leistungsanerkennung.

Prüfungsausschuss (PA)
Entscheidet über den Ablauf der Prüfungen, setzt Prüfungsordnung fest und erkennt bereits erworbene Prüfungsleistungen an. Für Quereinsteiger also eine wichtige Anlaufstelle. Zudem ist er für alle Arten von Anträgen zuständig.

Prüfungsordnung (PO)
Die vom PA festgelegten Regeln, nach denen Prüfungsleistungen erbracht, gewertet und berechnet werden.

Rechenzentrum (RZ)
Hier gibt es Lizenzen und Hilfe für diejenigen, die ihr Passwort vergessen haben.

Regelstudienzeit
In den Prüfungsordnungen angegebene, sehr optimistische Zeitspanne, in der das Studium absolviert werden soll. Unter anderem orientieren sich die BAföG-Bestimmungen an dieser Zeitspanne.

Rektor
Der Rektor ist der Vertreter der gesamten Uni gegenüber der Öffentlichkeit und dem Ministerium. Seit Dezember 2015 ist Prof. Dr. Axel Schölmerich im Amt.

Rekursion
Siehe ->Rekursion

Rub Internet Connector (rubicon)
Tool mit dem es manchmal möglich ist, auf diverse elektronische Dienste der Uni zuzugreifen (Studienbescheinigung, Semesterticket, VSPL)

Rückmeldung
Ein bürokratischer Akt, der jedes Semester innerhalb einer bestimmten Frist vorgenommen werden muss. Bei Versäumnis: Vorläufige Exmatrikulation

Semesterticket
Preisgünstiges Ticket, das in Verbindung mit dem Studierendenausweis jeweils für ein Semester zur Benutzung von öffentlichen Verkehrsmitteln berechtigt. Ist im Sozialbeitrag enthalten, kann sich jeder ohne Anmeldung im Foyer der Universitätsverwaltung abholen. Ab 19 Uhr und am Wochenende kann eine zweite Person mitgenommen werden.

Semesterwochenstunden (SWS)
Anzahl der Stunden, die im Laufe eines Semesters in jeder Woche auf Lehrveranstaltungen entfallen. Vor- und Nachbearbeitung sind darin nicht enthalten.

Senat
Wird einmal im Jahr bei den Gremienwahlen gewählt. Vorsitzender ist der Rektor. Der Senat war vor dem Hochschulrat das oberste beschlußfassende Gremium der Universität.

Skript
Schriftliche Ausarbeitung von Vorlesungen, werden manchmal von den Lehrstühlen ausgegeben.

Sozialbeitrag
Pro Semester zu leistende Zahlung, mit der verschiedene Dinge wie das Semesterticket und die Mensa finanziert werden, etwa 270€.

Stipendium
Studierende können bei verschiedenen Stiftungen Stipendien beantragen, deren Höchstgrenze meist über denen des BAFöG liegen und nicht an die Regelförderungszeit gebunden sind.

Studiendekanat
Koordiniert Verwaltungsabläufe des Studiengangs. Insbesondere findet hier auch die Studienberatung statt, was für dich am wichtigsten sein dürfte.

Studienkreis Film (SKF)
Einer der ältesten studentischen Filmclubs Deutschlands. Führt regelmäßig sehr günstig Filme in einem Hörsaal der Uni auf.

Studierendenparlament (StuPa)
Verfügt über 35 Sitze und wird einmal jährlich von allen an der Uni eingeschriebenen Studierenden gewählt. Zu seinen wichtigsten Aufgaben gehören die Wahl des AStA und die Genehmigung des Haushaltes. 

U35
Chronisch überlastete Straßenbahn, die UNI und Hauptbahnhof verbindet.

Uni-Sekretariat
Zuständig für Immatrikulation, Exmatrikulation, Rückmeldung, Beurlaubung etc. Du findest es in der Universitätsverwaltung.

Uni-Center
Auf der anderen Seite der Brücke gelegene Einkaufszone mit grimmigen Sicherheitskräften.

Universitäts-Bibliothek (UB)
In der Uni-Bibliothek darf sich jeder Student ohne weitere Anmeldung Bücher ausleihen. Der Studierendenausweis genügt hierzu. Sie ist zu finden in dem großen Gebäude zwischen Studierendenhaus und Audimax.

Vollversammlung (VV)
Der fromme Wunsch, möglichst viele Studenten in einem Raum anzusammeln. Dies geschieht einmal im Semester für die Fachschaft, um den Fachschaftsrat zu wählen und ihm seine Aufgaben zu geben.

VSPL
•	Für uns nicht verbindliches System zur elektronischen Kurs- und Prüfungsanmeldung
•	Noch vor W3L der Größte ProgrammierGAU an der Uni für mehrere Millionen Euro.

Wohnheim
Jeder eingeschriebene Student der Uni kann bei der AKAFÖ-Wohnheimverwaltung einen Antrag auf ein Wohnheimzimmer stellen.