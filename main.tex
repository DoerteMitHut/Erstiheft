% 
% Example of how to use a custom TTF font by following the directions here:
% http://math.stanford.edu/~jyzhao/latexfonts.php 
%
% In this case, the custom font is LoKinderSchrift, from
% http://www.fonts101.com/fonts/view/Uncategorized/22887/LoKinderSchrift
%
\documentclass[8pt, twocolumn]{book}

\usepackage[english]{babel}
\usepackage[utf8x]{inputenc}
\usepackage{helvet}
\renewcommand{\familydefault}{\sfdefault}
\usepackage{blindtext}
\usepackage{xcolor}
\usepackage{graphicx}
\usepackage{ifthen}
\usepackage[explicit]{titlesec}
\usepackage{pgffor}
\usepackage{tocloft}
\usepackage{fancyhdr}
\usepackage{xstring}
\usepackage{amssymb}

\newcommand\customfont[1]{{\usefont{T1}{custom}{m}{n} #1 }}
\newcommand{\makeDate}[2][addYear]{
    #2\ifthenelse{\equal{#1}{addYear}}
                  {\the\numexpr(\the\year) - 2000\relax}
                  {}
}
\usepackage{xstring}
\newcommand\IfStringInList[2]{\IfSubStr{,#2,}{,#1,}}


\titleformat{\section}[display]{\huge}{\thesection}{2em}{#1}%

\fancyhf{}
\cfoot{\thepage}






%==============Directories=============================
\def \imagePath{./images}% directory containing images
%==============title image=============================
\def \titleImageFilename{brinpage}% filename of title image without suffix
\def \titleImageDescription{Sergey Brin und Larry Page, Gründer von Google}
%==============space time coordinates==================
%==locations==
\def \notYetSet{\[/////\]}
\def \roomFreshmenBrunch{UFO 0/02}                                  % Raum für's Erstifrühstück
\def \initialSegregationMeetingPoint{\notYetSet}                    % Treffpunkt Tutorieneinteilung
\def \roomIntroductoryEvent{\notYetSet}                             % Raum der Einführungsveranstaltung
\def \PubcrawlMeetingPointOne{\notYetSet{}}                         % Erster Treffpunkt Kneipentour
%\def \PubcrawlMeetingPointTwo{Am Bochumer Hauptbahnhof (Haupthalle)}% Zweiter Treffpunkt Kneipentour
\def \PubcrawlStartingPub{Game am Bochumer Rathaus}                 % Erste Kneipentourkneipe
\def \excursionDestination{\notYetSet}                              % Ziel der Erstifahrt im Akkusativ für Zielangebende Präpositionen
%==dates and times== 
\def \currentTwoDigitYear{\the\numexpr(\the\year) - 2000\relax}
\def \firstDayOfFreshmenWeek{08.10.}
\def \firstDayOfExcursion{09.11.}



\begin{document}
\onecolumn
    \raisebox{-\totalheight}[0pt][0pt]{
        {\hspace{-2.25cm}\includegraphics[width=0.97\paperwidth, trim = 0 0 0 -1.5cm]{\imagePath/\titleImageFilename}}
    }
\begin{minipage}{\textwidth}    
    \vspace*{-3cm}
    \hspace{-\dimexpr\oddsidemargin+3cm}
    \colorbox{black}{
        \makebox{
            \hspace{1cm}\textcolor{white}{\fontsize{70}{5}\customfont{Readme.txt}}\hspace*{1cm}
        }
    }
    \noindent\hspace*{-1cm}
    \fbox{
        {\fontsize{18.5}{1}\textmd{ERSTI-INFO DER FACHSCHAFT ANGEWANDTE INFORMATIIK}}
    }\\
    \parbox[t][21.65cm][b]{\textwidth}{
    %\vspace*{18cm}
    \hspace*{-1.8cm}
    \fcolorbox{black}{black}{
        \parbox[t][1cm][c]{4cm}{
            \textcolor{white}{
                \fontsize{22}{1}\textmd{
                    WS \currentTwoDigitYear/\the\numexpr (\currentTwoDigitYear) + 1\relax
                }
            }
        }
    }
}
\vspace*{-10cm}
\end{minipage}
\newpage

\pagestyle{fancy}  

\twocolumn
\section*{Herzlich Willkommen}

In den ersten Wochen kommen eine Menge Informationen auf dich zu. Dieses Heft soll nicht nur ein Teil davon sein, sondern dir das Mitschreiben ersparen und dir helfen, den Überblick zu behalten.\\
Wichtige Termine am Anfang sind die Ersti-Woche und die Ersti-Fahrt. Die Ersti-Woche beginnt am \makeDate{\firstDayOfFreshmenWeek} und soll dir eine erste Orientierung im Studiengang geben. Hier wirst du von den Tutoren und dem Fachschaftsrat betreut.
Die Ersti-Fahrt findet am Wochenende vom \makeDate{\firstDayOfExcursion} bis zum  \makeDate{\lastDayOfExcursion} in \excursionDestination statt und soll es dir ermöglichen, Kontakte mit anderen Erstsemestern aber auch mit bereits fortgeschrittenen Studierenden zu knüpfen.\\

\textit{Der Fachschaftsrat}

\clearpage

%==============Directories=============================
\def \imagePath{./images}% directory containing images
%==============title image=============================
\def \titleImageFilename{brinpage}% filename of title image without suffix
\def \titleImageDescription{Sergey Brin und Larry Page, Gründer von Google}
%==============space time coordinates==================
%==locations==
\def \notYetSet{\[/////\]}
\def \roomFreshmenBrunch{UFO 0/02}                                  % Raum für's Erstifrühstück
\def \initialSegregationMeetingPoint{\notYetSet}                    % Treffpunkt Tutorieneinteilung
\def \roomIntroductoryEvent{\notYetSet}                             % Raum der Einführungsveranstaltung
\def \PubcrawlMeetingPointOne{\notYetSet{}}                         % Erster Treffpunkt Kneipentour
%\def \PubcrawlMeetingPointTwo{Am Bochumer Hauptbahnhof (Haupthalle)}% Zweiter Treffpunkt Kneipentour
\def \PubcrawlStartingPub{Game am Bochumer Rathaus}                 % Erste Kneipentourkneipe
\def \excursionDestination{\notYetSet}                              % Ziel der Erstifahrt im Akkusativ für Zielangebende Präpositionen
%==dates and times== 
\def \currentTwoDigitYear{\the\numexpr(\the\year) - 2000\relax}
\def \firstDayOfFreshmenWeek{08.10.}
\def \firstDayOfExcursion{09.11.}



Um dir den Einstieg in den Unialltag zu erleichtern, beginnt die erste Woche der Vorlesungszeit nicht direkt mit dem vollen Lernprogramm.

In der Ersti-Woche vom \makeDate{\firstDayOfFreshmenweek} bis zum \makeDate{\lastDayOfFreshmenweek} hast du die Gelegenheit, auf einfache Weise Leute aus deinem Studiengang kennen zu lernen. Schließlich wirst du mit diesen einen mehr oder minder großen Teil deines Studiums verbringen und zusammen macht es einfach mehr Spaß. Die Ersti-Woche dient auch dazu, dir ein paar Einblicke in das Uni-Leben zu geben und dir Dinge zu zeigen, auf die du sonst vielleicht gar nicht so ohne weiteres gestoßen wärest.\\

Für dich beginnt die Woche am Montag um \initialSegregationMeetingTime Uhr \initialSegregationMeetingPoint . Dort gibt es eine kurze Begrüßung und anschließend stellen sich die Tutoren vor. Diese teilen euch dann in Gruppen ein, die für die Dauer des ersten Semesters beibehalten werden sollten. Um \AudimaxAddressingStartingTime Uhr geht es dann weiter mit der zentralen Immatrikulationsfeier im Audimax. Dort versammeln sich alle Erstis der RUB und lauschen andächtig den Ansprachen von Rektor, Bürgermeister und Anderen um dann im Anschluss vom Fachschaftsrat auf dem Forum begrüßt zu werden.\\
Am Dienstag geht es um \startingTimeOfTutorProgram Uhr mit dem Tutorenprogramm weiter. Um \startingTimeOfOfficialWelcomingEvent Uhr beginnt die offizielle Einführungsveranstaltung für den Bachelor-/Master- Studiengang „Angewandte Informatik“ im \roomIntroductoryEvent. Noch wichtiger: Am Abend veranstalten wir eine gemeinsame Kneipentour in die Bochumer Innenstadt. Hier ist um \firstMeetingTimePubCrawl Uhr erster Treffpunkt \PubcrawlMeetingPointOne mit Wegbier, zweiter Treffpunkt um \secondMeetingTimePubCrawl Uhr \PubcrawlMeetingPointTwo und ab \startingTimePubCrawl Uhr beginnt \PubcrawlStartingPub die Kneipentour.\\

Am Mittwoch ist dann der Tag der Fachschaft und der Fachschaftsrat darf in Aktion treten. In \roomFreshmenBrunch werden wir zusammen ab \startingTimeFreshmenBrunch Uhr brunchen und uns vorstellen, damit du weißt, wen du bei allen auftretenden Fragen und Problemen ansprechen kannst. Wenn du Fragen – egal, welcher Art auch immer – haben solltest, zögere nicht sie auszusprechen, dafür sind wir ja schließlich da. Wir planen außerdem eine Campusrallye, die dir die wunderschönen Betonklötze näher bringen soll, die du die nächste Zeit täglich besuchen darfst. Die Gewinner erhalten tolle Geschenke.\\

Ab Freitag erwarten dich dann die ersten Vorlesungen und das Studieren geht richtig los! Um dich nach den ersten Wochen von dem ersten Schock zu erholen, laden wir dich ein mit uns gemeinsam \excursionDestination zu fahren. Dort wollen wir das Wochenende (\makeDate{\firstDayOfExcursion} - \makeDate{\lastDayOfExcursion}) ganz locker und vor allem mit Spaß genießen. Mehr Infos dazu & Anmeldemöglichkeiten gibt es schon am Tag der Fachschaft.\\

Wie du siehst, erwartet dich ein volles Programm. Aus Erfahrung lässt sich jedem Erstsemester nur raten, die Termine wahrzunehmen, um Kontakte zu knüpfen so wie Uni und Fachschaft kennen zu lernen.\\

Also, man sieht sich!
%\input{chapters/02-schedule.tex}
%Im Jahre 2004 entschied man sich an der RUB, auf den IT-Zug aufzuspringen. Dabei wollte man den Spagat zwischen der eher theoretischen Informatik, wie sie schon länger als Teil der Mathematik existiert, und den vielen Anwendungsfächern sowie Grundlagen aus anderen Fachbereichen, die einem nicht fehlen sollten, wagen. Diese breite Fächerbasis fasste man medienwirksam unter dem Begriff „polydisziplinär“ zusammen. 2013 dann ging die Leitung des Studiengangs von der Fakultät für Elektro- und Informationstechnik (ET/IT) auf das Institut für Neuroinformatik (INI) über, das viele Lehrveranstaltungen für den Studiengang anbietet.

In den ersten Semestern wirst du eine ganze Menge Grundlagen der Mathematik und Informatik kennenlernen. Aller Anfang ist schwer und so lehrt die Erfahrung, dass in der AI - wie in den meisten Studienfächern - die Zahl deiner Kommilitoninnen und Kommilitonen innerhalb der ersten Semester stark schrumpfen wird. Dabei sollten sich alle Zweifelnden bewusst sein, dass die oft trockenen Vorlesungen des Kernbereichs nach dem dritten Semester abnehmen und durch Veranstaltungen der selbst gewählten Vertiefungsfächer aus zahlreichen Wahlkatalogen ergänzt werden.

Solltest du zu irgendeinem Zeitpunkt Fragen zu deinem Studium haben, zögere nicht, deinen Tutor oder ein Mitglied des Fachschaftsrates anzusprechen und um Beistand und Information zu ersuchen.

Die Sitzungen des Fachschaftsrates finden im Abstand von einer Woche das ganze Semester über statt.
%\input{chapters/04-servicesAd.tex}
%Höhere Mathematik I
Für manche AI-ler ist dies die schwerste Vorlesung - andere hingegen haben damit weniger Probleme. Fakt ist, dass es sich bei Frau Kasco um eine sehr nette und hilfsbereite Dozentin handelt, die ihren Studenten gerne entgegenkommt. Gegen Ende des Semesters wird es zwei Probeklausuren geben, welche dir eine gute Gelegenheit geben, zu prüfen, wie gut du auf die Klausur vorbereitet bist. Da du dir in jeder Probeklausur bis zu 10 Bonuspunkte verdienen kannst, die sogar vor Bestehen der echten Klausur auf dein Endergebnis angerechnet werden, solltest du dich in jedem Fall gut auf die Probeklausuren vorbereiten! Du benötigst im Extremfall also bloß 30% in der Klausur, um die Prüfung zu bestehen.

Hinweis: Du darfst 5 beidseitig handbeschriebene Zettel in die Klausur mitnehmen.
Statistik
Lass dich nicht von dem einfachen Anfang dieser Vorlesung täuschen, denn Statistik wird mit der Zeit ziemlich verwirrend. Es mag zu Beginn noch einfach sein – Klassische Wahrscheinlichkeitsrechnung und Kombinatorik – später kommen Bernoulli-Verteilung, Ungleichung von Chebyshev, Signifikanztests... wer da nicht am Ball bleibt, weiß bald nicht mehr, wo oben und unten ist.

Hinweis: Formelsammlung und Skript werden vom Lehrstuhl verkauft (und online gestellt). In der Regel werden sie auch in der ersten Vorlesung angeboten, also sollest du Geld dabei haben und dir damit den Weg zum Lehrstuhl sparen. In der Klausur ist nur die offizielle (und ausgedruckte) Formelsammlung erlaubt.

Tipp: Statistik ist sehr unintuitiv, die offensichtliche Lösung ist oft falsch und die richtige Lösung manchmal nicht logisch nachvollziehbar. Also besser 2x nachdenken und im Zweifelsfall stur durchrechnen.

Informatik I
Während sich bis zum Wintersemester 14/15 der berühmtberüchtigte Professor Helmut Balzert die Ehre gab, die Erstsemester in den Künste der Programmierung zu unterweisen, übernimmt dies seit dem Wintersemester 15/16 Professor Hübner. Da der Lehrstuhl von Professor Hübner eher hardwareorientiert und maschinennah arbeitet, wird in dieser Veranstaltung die Programmiersprache C++ gelehrt, welche aufgrund ihrer manchmal schwer nachvollziehbaren Eigenheiten von Einsteigern oft als eher schwierig empfunden wird. 

Programmieren in C
Allen, die schon programmieren können, dürfte die Veranstaltung anfangs recht simpel erscheinen. Es geht um Datentypen, if- und while-Blöcke etc. Allerspätestens bei Arrays und Zeigern sollten aber alle aufpassen, denn hier haben viele Leute Probleme und C hat da so seine Eigenarten.

Tipp: Die Vorlesung wie auch die Klausur sind eher theoretisch ausgerichtet. Es genügt also nicht, programmieren zu können; man muss auch die Details von C kennen und genau wissen, wie die Sachen unter der Haube funktionieren.

Wirtschaftlichkeitsanalyse
Hier bewegst du dich im Bereich der BWL. Frau Wischermann wird sich nach Kräften bemühen, euch den Sinn der Kosten- und Investitionsrechnung ersichtlich zu machen. Wenn du Fragen hast, gibt es keinen Grund, sie nicht zu stellen. Sie wird gerne Sachen wiederholen, bis sie wirklich alle verstanden haben.

Tipp: Die vorgestellten Verfahren werden in der Klausur in kurzer Zeit abgefragt. Eine gewisse Routine ist von Vorteil und Aufgabenstellungen sollten genauestens gelesen werden. Im Zweifelsfall nachfragen.

Nichttechnisches Wahlfach:
Für das erste Semester ist neben den bereits genannten Fächern das so genannte Nichttechnische Wahlfach vorgesehen. Hier sollst du ein Fach belegen, mit dem du über den Tellerrand hinaus schaust. Auf der Webseite der AI findet sich eine (nicht abschließende!) Liste mit möglichen Fächern. Bei anderen Fächern solltest du bei der Studienberatung nachfragen. Mit Sprachen (nicht Programmiersprachen!) oder Sport bist du aber auf der sicheren Seite. Als Empfehlung können wir "Englisch für Angewandte Informatiker" nennen, welches genau die 5 CP gibt, die du für das Nichttechnische Wahlfach brauchst. Du kannst übrigens auch mehr als ein Fach wählen, falls du mit einem alleine nicht auf 5 CP kommst.

Hinweis: Das Nichttechnische Wahlfach hat seinen Weg durch eine Vorschrift ins erste Semester gefunden, die für Studienpläne ziemlich penibel 
30 CP/Semester vorschreibt. Das Nichttechnische Wahlfach ist erfahrungsgemäß in späteren Semestern besser angesiedelt, wo bereits Abweichungen vom Musterstudium erfolgt sind und sich vielleicht Lücken auftun, die es effizient zu schließen gilt. 
%Allgemeines
Als Student studiert man unter einer Prüfungsordnung (PO). Diese legt den Aufbau des Studiums und die „Spielregeln“ fest. Jeder, der sich ab dem WS13/14 in die Angewandte Informatik einschreibt, studiert unter der PO13. Die Grundzüge wollen wir dir hier etwas näher bringen, garantiert präzise Informationen gibt es aber nur im Originaltext der PO.
www.ai.rub.de/studierende/ordnungen/po13.html

Eckdaten und Credit Points
Die Regelstudienzeit beträgt 6 Semester, die Erfahrung zeigt jedoch, dass nur Wenige diesen strikten Zeitplan einhalten können. Ein oder zwei Semester an den Bachelor anzuhängen, ist jedoch kein Weltuntergang, da man sowohl zum Winter- als auch zum Sommersemester in den Master eingesteigen kann. In dieser Zeit gilt es, mindestens 180 Credit Points (CP) zu sammeln, wobei ein CP – so die Idee – etwa 30 Arbeitsstunden im Semester entsprechen sollte, wobei darin sowohl Anwesenheit in Vorlesungen als auch das Pauken zu Hause enthalten sind.
Damit kannst du dir leicht ausrechnen, dass du in jedem Semester ca. 30 CP erlangen solltest und dementsprechend ist das Studium auch aufgebaut. Diese Rechnung geht davon aus, dass du immer alles bestehst, was oft nicht der Fall ist.

Prüfungen
CP sammelst du durch das Bestehen von Veranstaltungen, aus denen sich das Studium zusammensetzt. Eine Veranstaltung besteht in den meisten Fällen aus wöchentlichen Vorlesungen und Übungen und deckt einen Themenkomplex ab. In der Vorlesung werden dabei Verfahren und Methoden vorgestellt und erklärt, die du dann in der Übung bzw. beim Pauken selbstständig anwenden sollst.
Für jede Veranstaltung, die du bestehen willst, musst du eine Prüfungsleistung erbringen. Dies ist in der Regel eine Klausur am Ende des Semesters, manchmal aber auch eine mündliche Prüfung, eine Präsentation oder die Abgabe von Aufgaben während der Vorlesungszeit. Die meisten Fächer im Studium werden benotet, und zwar mit Prozentpunkten zwischen 0 und 100, wobei 50% für ein Bestehen erforderlich sind. Bei reinen Multiple-Choice Klausuren können abweichende Kriterien anfallen. Bei allen anderen Fächern genügt es, sie mit mindestens 50% zu bestehen. Hier spricht man von einem Leistungsnachweis.
Klausuren sollen nach 4 Wochen bewertet sein, worauf man allerdings manchmal vergebens hofft. Wartezeiten von mehreren Monaten kommen manchmal vor. Des Weiteren hast du das Recht auf eine Einsicht, d.h. nachdem die Klausur bewertet wurde,  kannst du sie noch mal ansehen und auf eventuelle Mängel in der Bewertung hinweisen. Auch hier steht die Realität noch weit hinter der Idee zurück, manche Dozenten bieten gar keinen Termin für die Einsicht an und lassen dich nur nach Terminvereinbarung die Klausur einsehen.

Nichtbestehen und Rauswurf
Nicht bestandene Prüfungen müssen wiederholt werden, wobei (fast) jede Prüfung 2x im Jahr angeboten wird. Du solltest es aber gar nicht erst dazu kommen lassen, denn du hast - im Gegensatz zur PO09 - pro Prüfung nur 3 Versuche, also 2 Wiederholungsversuche. Fällst du also einmal durch, sollte dich das nicht direkt entmutigen, aber halte dich ran, denn sonst hast du im nächsten Semester noch mehr Arbeit. Wenn du 3x durchfällst, wird dir die Uni ein „endgültiges Nichtbestehen“ (ENB) bescheinigen, was das Ende deines Studiums (und i.d.R. aller anderer Informatik-verwandter Studiengänge) bedeutet.

Verbesserungsversuche
Deine Note passt dir nicht? Im Vergleich zu deinen Vorgängern bist du da in einer besseren Lage: Seit dem WS 15/16 darfst du nicht nur für drei, sondern für fünf Prüfungen einen Verbesserungsversuch in Angriff nehmen, wobei das beste Ergebnis zählt. Das gilt nur für bestande Prüfungen. Um das 3-Versuche-Limit kommst du damit also nicht herum!

An- und Abmeldung zu Prüfungen
Zu Prüfungen muss man angemeldet sein! Im ersten und zweiten Semester wirst du für alle Pflichtprüfungen automatisch angemeldet. Für Wahlpflichtfächer und das Nichttechnische Wahlfach musst du dich eigenständig anmelden. Das Gleiche gilt für Pflichtfächer ab dem dritten Semester. Dies tust du entweder beim Prüfungsamt oder online über die Plattform FlexNow. Bist du für eine Prüfung angemeldet und bestehst sie nicht, wirst du automatisch für die Wiederholungsprüfung zum nächstmöglichen Termin angemeldet.
Von Prüfungen kann man sich auch abmelden, allerdings nicht im ersten Semester. Im Zweiten ist ein Beratungsgespräch dafür erforderlich. Danach kannst du dich selbstständig abmelden. Allerdings ist dies für jede Prüfung nur 3x möglich. Für den nächstmöglichen Termin wirst du dann wieder angemeldet. Außerdem kannst du bis zu drei von dir gewählte Vertiefungsmodule durch andere Vertiefungsmodule austauschen, allerdings nicht nachdem du endgültig nicht bestanden hast (3x durchgefallen) und auch nicht, nachdem du die neue Prüfung bereits geschrieben hast. Solltest du krankheitsbedingt nicht zu einer Prüfung erscheinen können, ist ein ärztliches Attest erforderlich, damit dir kein Versuch abgezogen wird!

Vertiefungsbereich und Wahlfächer
Darüber, welche Module du bestehen musst, gibt das Modulhandbuch Aufschluss. Zusätzlich zu den Pflichtmodulen gibt es noch den Vertiefungsbereich, der mit Modulen im Gesamtumfang von 30 CP gefüllt werden will. Hierzu kannst du Veranstaltungen aus mehreren Katalogen frei wählen. Die verschiedenen Veranstaltungen solltest du im Zeitraum vom 3. bis zum 6. Semester belegen. Vorher gibt es aber noch die „Nichttechnischen Wahlfächer“, die du namensgerecht frei belegst, sie sollten zusammen 5 CP umfassen und  theoretisch im 1. Semester absolviert werden.  Allerdings kommen die Wenigsten im 1.  Semester dazu.
Später im Studium hast du dann noch ein Studienprojekt, ein Seminar und am Ende die Bachelorarbeit hinter dich zu bringen.

Dein Abschluss
Wenn du nun alle Module bestanden und 180 CP erreicht hast, erhältst du ein Bachelorzeugnis, in dem deine Durchschnittsnote, die Noten der einzelnen Module und sowohl das Thema als auch die Note deiner Bachelorarbeit festgehalten werden. Ferner bekommst du eine Bachelorurkunde und darfst dich „Bachelor of Science“ und „Ingenieur“ nennen.
Ab in den Master!
Lust auf mehr? Für 120 weitere CP gibt's den „Master of Science“, der auf 4 Semester ausgelegt ist. Bis dahin ist es noch ein weiter Weg. Doch bereits jetzt solltest du wissen, dass die Masterstudienplätze nach einem NC-Verfahren vergeben werden. Es ist also sehr ratsam, sich eine gute Durchschnittsnote im Bachelor zu erarbeiten, da man so die Chancen auf einen Masterstudienplatz maximiert.
%Dieses Kapitel richtet sich hauptsächlich an die Masterstudenten, die ihren Bachelor woanders gemacht haben, aber vielleicht finden auch die bisherigen AI-ler ein paar nützliche Informationen. Abgesehen von den bachelorspezifischen Sachen in diesem Erstiheft sind die meisten Informationen auch für Masterstudenten nützlich. Deswegen folgt hier eine kleine Zusammenfassung der verschiedenen Studienmöglichkeiten, die man hier als Masterstudent hat.

Pflichtveranstaltungen
Da wäre zunächst das Fehlen von Pflichveranstaltungen. In früheren Prüfungsordnungen gab es davon nur zwei, jetzt kann man frei wählen, auch wenn im Wahlpflichtmodul (20CP) die Auswahl etwas enger ist. Welche Veranstaltungen man aus diesem Modul nimmt, ist natürlich Geschmackssache. Wer an Theoretischer Informatik Spaß hatte, dem dürfte Komplexitätstheorie sicher gefallen. Effiziente Algorithmen wird von Frau Kacso gehalten, die keinen Studenten sitzen lässt und auch gerne in ihrer Sprechstunde alles lang und breit noch einmal erklärt.
Groupware und Wissensmanagement ist eine ganz andere Art von Vorlesung, das geht schon viel mehr in Richtung Geisteswissenschaften. Wenn man keine Klausuren mag, ist man hier richtig, denn benotet wird das Fach nach mehreren Projektaufgaben, die man im Team bearbeitet.
Supervised- und Unsupervised Methods wechseln schonmal das Vortrags- und Prüfungsformat,  was allerdings dem Inhalt keinen Abbruch tut. Die Materie wird systematisch und nachvollziehbar erarbeitet und beide Dozenten verstehen sich darauf, auf den einzelnen einzugehen und Fragen zu beantworten.
Parallel Computing und Nebenläufige Programmierung behandeln ähnliche Inhalte und prüfen im Falle des ersteren im Rahmen einer Projektarbeit , im Falle des letzteren im Rahmen einer Klausur. Nebenläufige Programmierung erntet häufig Kritik am Dozenten, soll Inhaltlich jedoch recht lohnenswert sein, wenn man sich penibel an die Vorgaben aus der Vorlesung hält. 



Vertiefungsmodule
Die Vertiefunsgmodule sind die eigentlich interessanten Fächer. Hier kann man sich aus einem Katalog zwischen einigen verschiedenen Bereichen Vorlesungen aussuchen.

Ingenieurinformatik
In der Ingenieurinformatik findet man unter anderem Fächer, die den Bauingenieuren nahe liegen, aber auch allgemeinere Fächer wie Product Lifecycle Management oder IT im Engineering.

Neuroinformatik
Die Neuroinformatikfächer beschäftigen sich mit maschinellen Lernverfahren, Computersehen und autonomer Robotik. Die Kurse hier sind meist recht mathematiklastig und die Gruppen dementsprechend recht klein, aber sie lohnen sich wirklich, und die Neuroinformatiker bieten meist sehr interessante Masterarbeiten und Studienprojekte an. Lasst euch von Professor Wiskotts teilweise wirklich schwierigen Aufgabenzetteln nicht erschrecken, denn die Prüfungen sind im Verhältnis zu den Übungsaufgaben recht leicht. Professor Wiskott geht auch gerne mit den Studenten nach der Vorlesung in der Mensa essen, das sollte man sich nicht entgehen lassen.

Kryptologie und TI
Zu Kryptologie und Theoretische lässt sich sagen, dass es sehr mathematiklastig ist. Das bedeutet hier zum Teil weniger Rechnen als Beweisen. Trotzdem (oder gerade deswegen) können diese Vorlesungen wirklich interessant werden, und um Mathematik kommt man als Informatiker sowieso nicht herum, zumindest nicht, wenn man es richtig macht.

Operations Research & Management
Operations Research... ist BWL. BWL-Bashing ist unter nicht wenigen [AI]-lern so eine Art Freizeitvergnügen: wenn man nichts besseres zu bereden hat, lästert man über die "BWLer". Wenn euch die Vorlesung anspricht, oder euch Einführung Management Science gefallen hat, dann lasst euch dennoch nicht davon abhalten, sie zu hören.

Bioinformatik
Die Bioinformatik-Sektion ist neu im Modulhandbuch und es existieren entsprechend spärliche Erfahrungswerte. Probiert's aus und sagt uns Bescheid.

Weitere Veranstaltungen
Bleiben noch die Seminare (sucht euch irgendwas Interessantes aus! Man kann auch Seminare machen, die nicht in der Liste stehen, solange diese vom Thema her passen), die Freien Wahlfächer (irgendwas, was man schon immer mal machen wollte, von Astrophysik bis Zahnmedizin), das Studienprojekt (im Prinzip wie im Bachelor, nur vom Umfang und Anforderung höher) und die Masterarbeit.

Für Letztere sollte man viel Zeit einplanen, normalerweise ist sie die einzige Veranstaltung, die man im Semester hat. Überlegt euch am besten schon früh, an welchem Lehrstuhl oder zu welchem Thema ihr die Arbeit machen wollt, fragt rechtzeitig nach und lernt die Leute ein bisschen kennen, die dort am Lehrstuhl arbeiten.

Das war es zu den Wahlmöglichkeiten. Ansonsten kann man hier eigentlich nur die Tipps für Bachelor wiederholen. Lernt Leute kennen, knüpft Kontakte, seid sozial (auch wenn sich das als Klischeeinformatiker vielleicht komisch anhört), denn es hilft immer, ein paar Leute zu kennen. Man kann zusammen lernen, sich gegenseitig mit Vorlesungsmitschriften aushelfen, einander an wichtige Termine erinnern, oh, und natürlich auch nicht-Uni-Kram machen.
%Unsere Fachschaft betreibt eine Dreifaltigkeit aus Blog, Forum und Cloud, die dem Austausch unter den Studenten dient. in der Cloud finden sich einige Materialien zu verschiedenen Veranstaltungen und der Fachschaftsrat freut sich auch über eure Beiträge und Mitschriften.
	

  Bilde mit anderen Studenten Lerngruppen. Sich gemeinsam durch schwierige Themen zu kämpfen hilft und stärkt (oder verdirbt) den Charakter. Hier sammelst du bereits erste Erfahrungen für spätere Gruppenarbeiten wie z.B. das Studienprojekt. Aber vergiss nicht: nur weil du in einer Lerngruppe bist, heißt es nicht, dass du dich von den Anderen abgrenzen solltest, denn das Studium verläuft meist nicht geradlinig und so kann es schnell passieren, dass du im vierten Semester in einigen Fächern alleine sitzt.

Den Bachelor in 6 Semestern zu machen ist machbar aber stressig. Wenn das BAföG nichtmer fließt, lässt sich als Informatiker prima nebenher arbeiten und  ein, zwei Semester mehr zu brauchen ist weit verbreitet. 
5 Verbesserungsversuche ermöglichen es, Prioritäten zu s.etzen
Du solltest dich auch schon in den ersten Semestern bei den Klausuren um gute Noten bemühen. Auch wenn dir gute Noten nicht besonders wichtig sind, solltest du folgendes beachten: Während man früher noch mit der Durchschnittsnote „befriedigend“ (ab 62%) einen sicheren Master-Studienplatz hatte, hat der Master nun einen NC.

Vielleicht hast du schon irgendwo anders studiert und dort Scheine erworben? Dann kannst du dir möglicherweise einige Fächer anerkennen lassen. Wenn deine Noten entsprechend gut sind, dann nimm die Chance wahr. Es könnten die Punkte sein, die dir ein Semester ersparen. 

Über  das “Microsoft Imagine” Programm (früher Dreamspark, noch früher MSDN Academic Alliance) können viele der Softwareprodukte aus dem Hause Microsoft zu Studienzwecken gratis bezogen werden.
https://msdnaa.ruhr-uni-bochum.de

Falls du den Speicherplatz deines RUB-E-Mail-Postfachs von 1GB auf 10GB erhöhen möchtest, kannst du dies eigenständig mit Hilfe des Selfcare-Interfaces tun. Dieses findest du unter: https://mail.ruhr-uni-bochum.de/mail/faq/selfcare

Erhöhte Koffeinzufuhr ist während des Studiums nicht auszuschließen. Glücklicherweise gibt es auf dem Campus unzählige Möglichkeiten, an neuen Stoff zu kommen. Zum einen gibt es in vielen Gebäuden Cafeterien (HZO, IB, ID, NA, NC, MA, Mensa-Foyer, Studierendenhaus, G-Reihe). Zum anderen bietet die Mensa (täglich geöffnet von 11:00 – 14:30 Uhr, Freitags bis 14:00) eine Vielzahl von Getränken und variierenden Speisen an. Wenn's mal etwas länger dauert, kannst du auch bis 16:00 Uhr das Bistro besuchen. 
Ein wenig edler und entsprechend kostspieliger ist ein Besuch im QWest auf der G-Seite der Universität. Hier gibt es von 9:00 bis 11:00 Uhr Frühstück, von 11:30 bis 14:30 Uhr Mittagessen, anschließend bis 18:00 Uhr Kaffee und daraufhin bis 22:00 Uhr Abendessen.
Nähere Infos zu all den genannten Einrichtungen findest du unter: http://www.akafoe.de/
Darüber hinaus gibt es noch das Uni-Center, wo du dich auch samstags oder zu späteren Uhrzeiten mit Speis und Trank
eindecken kannst.


Ungestörter Ort zum Lernen allein oder in der Gruppe gesucht? Die Uni-Bibliothek bietet hierfür einen speziellen Bereich an, in dem es nicht immer so totenstill wie im Rest der Bib sein muss.
Auch in der Mensa kann außerhalb der Betriebszeiten gepaukt werden. Von 9 bis 11 Uhr so wie von 15 bis 19:30 (Freitags bis 18 Uhr) ist der Eingang an der N-Reihe (Ostseite) geöffnet, über das Bistro oben kommt man nicht rein.
Weniger bekannt ist ein Raum im NB auf Ebene 03. Hinter einer erstmal abschreckenden Feuerschutztür findet man einen Raum mit vielen Whiteboards.
%Auf den nachfolgenden Seiten erhältst du einen kleinen Überblick über die Struktur der Uni und einige wichtige Einrichtungen, die dir deine Zeit etwas erträglicher gestalten können.

---------------------------------------------------------

Die Struktur der Uni...
Im Allgemeinen gibt es an der Uni vier Interessengruppen, ohne die der Betrieb nicht möglich wäre. Die größte Gruppe sind wir, die Studenten, mit ca. 43 000 Mitgliedern. Daneben gibt es noch die Gruppe Professoren (ca. 400) und wissenschaftliche Mitarbeitern (2000) und ca. 2000 Angestellte in Technik und Verwaltung.

Organisatorisch ist die Uni recht hierarchisch aufgebaut. Die Angewandte Informatik ist einer von etwas über hundert Studiengängen an der RUB. Auf der Ebene des Studiengangs AI gibt es Gremien wie den Prüfungsausschuss und die Qualitätsverbesserungskommission, die sich um die Belange des Studiengangs kümmern. Dort haben die Studenten Mitspracherecht. Die Vertreter entstammen dabei der Fachschaft, die auf ihrer Vollversammlung einen Fachschaftsrat wählt,  der wiederum neben anderen Aufgaben, z.B. der Organisation der Erstiwoche, Mitglieder für die Gremien des Studiengangs auswählt.

... und des Studiengangs
Jeder Studiengang gehört zu einer Fakultät. Jeder Studiengang? Nein! Ein von unbeugsamen [AI]-lern bevölkerter Studiengang z.B. wird derzeit von einem Institut geleitet. In unserem Falle handelt es sich um das Institut für Neuroinformatik (INI), das seit dem Wintersemester 2013/2014 offiziell die Federführung des Studiengangs übernommen hat. Damit geht einher, dass Vieles bei der AI anders geregelt ist als bei den meisten übrigen Studiengängen. Die Leitung unseres Studiengangs obliegt jedoch wie bei allen anderen Studiengängen einem Dekan, welcher aus der Gruppe der Professoren stammt. Unser derzeitiger Dekan ist Prof. Dr. Laurenz Wiskott. Über dem Dekan stehen auf universitärer Ebene der Senat, der Hochschulrat und das Rektorat.

Auf der Ebene der studentischen Selbstverwaltung ist die zentrale Instanz das Studierendenparlament (kurz StuPa), das jährlich den AStA (Allgemeiner Studierendenausschuss) wählt. Der AStA als "Regierung" des StuPa, wiederum bildet Referate und Arbeitsgruppen für verschiedene Aufgaben.

Des Weiteren gibt es noch viele kleinere Beiräte und Kommissionen die z.B. die Einführung von neuen Studiengängen vorbereiten oder Satzung der Bibliothek ändern. Daneben gibt es dann natürlich noch den Alltagsbetrieb des Studierens und Forschens. Hier sind die wichtigsten Stationen neben der Fachschaft, das Studiendekanat und das Prüfungsamt. Verwaltet werden die Studenten im Studierendensekretariat. Außerdem gibt es Bibliotheken auf Fakultäts-, und Institutsebene aber auch die zentrale Universitätsbibliothek. Daneben existiert außerdem das Rechenzentrum, das für die elektronische Infrastruktur und, auf dezentraler Ebene, für die CIP-Inseln zuständig ist.

Nicht zu vergessen ist auch das Studierendenwerk (AKAFÖ), das sich mit den Cafeterien und der Mensa um eure Verpflegung kümmert, die Wohnheime verwaltet und eure BAföG-Anträge bearbeitet.



Hochschulpolitik
Rektorat
Das Rektorat, bestehend aus dem Rektor (Prof. Dr. Axel Schölmerich), der Kanzlerin (Dr. Christina Reinhardt) und den drei Prorektoren (Forschung, Transfer und Nachwuchs: Prof. Dr.-Ing. Andreas Ostendorf, Lehre und Internationales: Prof. Dr. Kornelia Freitag, Planung und Struktur: Prof. Dr. Uta Hohn), leitet die Universität im Alltag.

Hochschulrat
Seit dem Jahre 2008 besitzt jede Universität in NRW einen Hochschulrat. Dieser wählt die Mitglieder des Rektorats, beaufsichtigt das durch die Hochschulleitung erledigte operative Geschäft, nimmt Stellung zu Rechenschafts- und Evaluationsberichten und hat darüber hinaus beratende Funktion. Außerdem muss dem Hochschulentwicklungsplan und dem Wirtschaftsplan durch den Hochschulrat zugestimmt werden.

Senat
Der Senat setzt sich aus dem Rektor und 25 gewählten Mitgliedern zusammen, die in folgende Gruppen eingeteilt sind:

•	Professoren (13 Mitglieder)
•	Wissenschaftliche Mitarbeiter(4 Mitglieder)
•	Mitarbeiter in Technik + Verwaltung (4 Mitglieder)
•	Studenten (4 Mitglieder)

Der Senat hat ein weitgestreutes Aufgabenfeld. Zum einen bestätigt er die Wahl der Mitglieder des Rektorats, zum anderen kann er diverse Ordnungen erlassen und ändern sowie über eine Menge andere Dinge Empfehlungen aussprechen und Stellungnahmen geben.

Die Mehrheitsverhältnisse sind leider recht unfair verteilt. Trotzdem solltet ihr euer Wahlrecht wahrnehmen und jährlich (meist im Juni) die studentischen Mitglieder mitwählen. Der Senat tagt monatlich öffentlich im Senatssitzungssaal in der Universitätsverwaltung (UV).


Studierendensekretariat
Das Studierendensekretariat verwaltet alle Studenten an der Universität. Falls du vor hast deinen Studiengang zu wechseln, dich vom Bachelor in den Master umzuschreiben oder du einfach nur eine Studienbescheinigungfiref benötigst, ist das Studierendensekretariat die richtige Adresse. Du findest es im im Gebäude SSC (Studierenden-Service-Center) 0/229. Öffnungszeiten: Mo – Fr: 9 – 12 Uhr; Mo, Mi, Do: 13:30 – 15 Uhr

Dekan
Unser aktueller Dekan ist Herr Prof. Dr. Laurenz Wiskott. Der Dekan ist das Oberhaupt des Studienganges. Der Fachschaftsrat steht in engem Kontakt zum Dekan und vertritt ihm gegenüber die Interessen der Studenten.

Studiendekanat
Die Leiterin des Studiendekanates, Frau Kallweit, sitzt im NB-Gebäude im Raum 02/72. Sarah Thiele ist studentische Mitarbeiterin im Dekanat und leistet die studentische Studienberatung. Zum Studiendekanat geht man, wenn man die Studienberatung in Anspruch nehmen möchte oder allgemeinen Fragen zum Studiengang oder der Prüfungsordnung hat. Offene Fragen bezüglich des Studienganges können in der Regel schnell beantwortet werden. 
Sprechzeiten: Di 12-14 Uhr, Do: 9 – 11 Uhr.

Fakultätsrat
Der Fakultätsrat entscheidet über die Details der Studiengänge, schlägt neue ProfessorInnen vor und entscheidet in letzter Instanz über die Belange der Studiengänge, wie die Prüfungsordnungen oder auch die Verwendung der Studiengebühren. Die Mehrheitsverhältnisse sind ähnlich denen im Senat.

Prüfungsausschuss (PA)
Im Prüfungsausschuss sitzen ebenfalls Professorinnen und Professoren, wissenschaftliche Mitarbeiterinnen und Mitarbeiter sowie Studenten. Im Prüfungsausschuss wird neben dem 
Hauptthema „Organisation von Prüfungen“, auch über die Anerkennung von Prüfungsleistungen, sowie über den möglichen Austausch von Studienfächern gesprochen. An den Prüfungsausschuss 
richtet ihr alle Anträge z.B. zur Anerkennung von Prüfungsleistungen, Härtefallanträge bei drei nicht bestandenen Versuchen in einer Prüfung, nach einem Fachwechsel oder beim Einstieg in den Master. Der Vorsitzende des Prüfungsausschusses ist derzeit Prof. Dr. Markus König von den Bauingenieuren.

Prüfungsamt
Das Prüfungsamt ist die erste Anlaufstelle für Prüfungsangelegenheiten. Bei Frau Nawrat und Frau Füllbeck könnt ihr euch für Prüfungen anmelden (innerhalb der so genanten Anmeldefrist) oder abmelden (mindestens zwei Wochen vor der Prüfung, besser früher!) und Auszüge mit der Übersicht über die erreichten Studienleistungen und Scheine erhalten. Inzwischen können die Klausuranmeldungen auch online über das „FlexNow“-Portal vorgenommen werden. Das Prüfungsamt befindet sich im GA 03/41.
Sprechzeiten: Mo: 10 – 12 Uhr, Mi: 11-13 Uhr

Qualitätsverbesserungskommission (QVK)
Die Zuteilung der Qualitätsverbesserungsmittel an die verschiedenen Fakultäten und Dozenten obliegt der Qualitätsverbesserungskommission. Auch hier haben studentische Vertreter ein Mitspracherecht, welches durch vom FSR entsandte Vertreter(innen) in Anspruch genommen wird.

Die Fachschaft (FS)
Die Fachschaft sind alle Studenten im entsprechenden Studienfach. Also auch du! Meist bilden sich deine Lern- und Projektgruppen aus der Fachschaft. Durch gemeinsame Veranstaltungen wie die Kneipentour, die Fachschaftsfahrt und die Weihnachtsfeier wird das Zusammengehörigkeitsgefühl der AI-ler gestärkt.

Die Vollversammlung (VV)
Auch wenn ihr nicht vorhabt, den Studiengang aktiv mit zu gestalten, sondern „einfach nur studieren“ wollt, solltet ihr trotzdem diese eine Veranstaltung auf jeden Fall besuchen: Die VV der Fachschaft, die am Anfang eines jeden Semesters stattfindet. Hier werden wichtige Informationen über den Studiengang mitgeteilt und wichtige Entscheidungen gefällt. Darüber hinaus wählt die Vollversammlung den Fachschaftsrat und beauftragt ihn mit Arbeitsaufträgen, die er im Laufe des anschließenden Semesters zu erfüllen hat. Es ist also die ideale Gelegenheit, Verbesserungsvorschläge für den Studiengang einzubringen und zu lösende Probleme anzusprechen. Gerüchteweise soll es  auf der VV als weiteren Anreiz auch Kekse und Bier geben!

Der Fachschaftsrat (FSR)
Der FSR ist ein auf der VV gewähltes Gremium und untersteht der Fachschaft. Die Mitglieder wollen das umsetzen, was die VV ihnen an Arbeitsaufträgen auferlegt hat. Der FSR vertritt die Fachschaft gegenüber Gremien wie dem Prüfungsausschuss oder dem Dekan und setzt sich für die Interessen der Studenten ein. Solltest du Fragen bezüglich deines Studiums haben, kann dir der FSR helfen oder dich an eine fachkundige Person weitervermitteln. Jeder Student der Angewandten Informatik kann an der Fachschaftsratssitzung teilnehmen und ist stimmberechtigt.

FachschaftsvertreterInnenkonferenz (FSVK)
Die FachschaftsvertreterInnenkonferenz ist ein Gremium dass niemals außen tagt. Hier treffen sich Vertreterinnen und Vertreter der einzelnen Fachschaften (bzw. meist aus den Fachschaftsräten), um sich gegenseitig über die aktuelle Lage ihres Studiengangs zu informieren, um die gemeinsame Arbeit zu koordinieren oder auch um der studentischen Senatsfraktion ihr Votum mitzuteilen.

Studierendenparlament
Das Studierendenparlament (StuPa) ist das höchte Gremium in der studentischen Selbstverwaltung. Hier wird einmal jährlich der AStA gewählt, der Haushalt geprüft oder auch Entscheidungen zum Semesterticket gefällt. Das Studierendenparlament besteht aus 35 Mitgliedern die verschiedenen Listen angehören. Die Wahlen zum Studierendenparlament finden jährlich am Ende des Wintersemesters statt. Da die Wahlbeteiligung bisher meist sehr gering war, seid ihr aufgefordert das zu ändern. Die Stimme des Studierendenparlaments hat nämlich nur dann ein hohes Gewicht, wenn es von ausreichend Studenten legitimiert ist.

AStA
Der Allgemeine Studierendenausschuss wird vom StuPa gewählt. Der AStA verfügt über das Geld der Studentenschaft. Momentan geht ein Teil eures Semesterbeitrags an den AStA, der damit verschiedene Veranstaltungen finanziert, aber auch eine Rechts-, AusländerInnen- und BaföG- Beratung anbietet. Außerdem unterhält der AStA das Kulturcafe, in dem häufig Veranstaltungen und Parties stattfinden sowie zwei Copyshops (in GA 03 und GB 02). Der AStA vertritt die Studentenschaft gegenüber der Öffentlichkeit. Im AStA-Flur im Studierendenhaus sind die verschiedenen Referate angesiedelt, in denen ihr auch Informationen und Beratung zu wichtigen Dingen des Studi-Alltags erhaltet (Finanzen, Wohnungssuche usw.). Übrigens stellt der AStA auch den Internationalen Studierendenausweis aus, der euch in vielen Ländern weltweit Vergünstigungen bringt.

AKAFÖ
Das akademische Förderungswerk kümmert sich um die wichtigen Details des Studentenlebens. Es betreibt die Mensen und Cafeterien auf dem Campus. Daneben ist es noch für die Wohnheime und das Bafög zuständig. Die Verwaltung des AKAFÖs und das BAföG-Amt findest du im Studierendenhaus. Boskop wird übrigens auch vom AKAFÖ finanziert. Das Akafö finanziert sich zum Teil durch den Sozialbeitrag. 105 € davon fließen dorthin.

CIP-Insel
In den mit CIP-Pool gekennzeichneten Räumen im Gebäude ID stehen Studenten der Angewandten Informatik Rechner zur freien Verfügung. Den entsprechenden Account bekommt man unter Vorlage einer gültigen Studienbescheinigung vor Ort. Die Tutoren werden in der ersten Woche die Anmeldung mit euch vornehmen. Die CIP-Insel hat meist von 10-18 Uhr geöffnet.

Rechenzentrum
Das Rechenzentrum stellt das informationstechnische Herz der Uni dar. Interessante Aspekte sind vor allem der Internetzugang auf dem Campus (per WLAN oder HIRN Port), der Download von campuslizensierter Software (z.B. Sophos Antivirus, Windows) und der Erwerb bzw. das Leasing von Laptops zu rabattierten Preisen.

Ausführliche Informationen findet ihr auf der Homepage: https://www.rz.ruhr-uni-bochum.de

Bibliothek
Die Bibliothek ist sehr zentral auf dem Campus angesiedelt (wenn ihr von der Uni-Brücke geradeaus in Richtung Uni lauft, landet ihr quasi direkt vor dem Gebäude). 

In der Bibliothek findet ihr jede Menge Bücher, Zeitschriften, Dissertationen, etc. Um etwas auszuleihen, braucht ihr lediglich euren Studierendenausweis. 

Da ihr die Bibliothek weder mit Taschen noch mit Jacken betreten dürft, empfiehlt sich die Mitnahme einer 2-Euro-Münze zwecks Anmietung eines Spindfachs, denn die Bibliothek wechselt nicht.

Im Foyer befindet sich außerdem noch ein Café, eine Etage tiefer Toiletten und innerhalb der Bibliothek einige Computerarbeitsplätze zur Buchrecherche.

Jeden Mittwoch um 12:15 Uhr gibt es eine Bibliotheksführung. 

Öffnungszeiten: Mo – Fr: 8:00-24:00 Uhr, Sa: 11:00-20:00 Uhr & So: 11:00-18:00 Uhr (ab 22 Uhr und Sonntags ist jedoch kein Servicepersonal anwesend)

Für weitere Infos siehe Kapitel „Links“.
%BAföG
„BAföG“ steht für Bundes-Ausbildungsförderungs-Gesetz. Dahinter verbirgt sich unter anderem eine Möglichkeit zur Studienfinanzierung für Studenten mit geringem Einkommen und Vermögen. Die gesetzlichen Hintergründe und Vorschriften sind zu komplex, um sie im Rahmen dieses Heftes wiederzugeben, zumal für nahezu jeden Studierenden irgendwelche Ausnahmen und Sonderregelungen greifen. Deshalb nur die beiden wichtigsten Aussagen:

BAföG wird nicht rückwirkend gezahlt! Bzw. nur rückwirkend bis zu dem Monat in dem du den Antrag eingereicht hast.

Da wird dir geholfen: Wer keinen hilfsbereiten BAföG Berater beim Akafö erwischt und Hilfe braucht, sollte die BAföG-Beratung des AStA aufsuchen und sich dort kompetent beraten lassen!
Oder googlen :)

Stipendien
Viele Organisationen haben es sich zur Aufgabe gemacht, Studierenden mit Stipendien zu fördern. Da dies nur wenige Studierende in Anspruch nehmen, lohnt es sich auf jeden Fall, ein Stipendium zu beantragen.

Grundsätzlich fördern die meisten Stiftungen analog dem BAföG-Satz (aber man muss eben im Gegensatz zum BAföG später nichts zurückzahlen). Darüber hinaus gibt es i.A. eine „ideelle“ Förderung in Form von Büchergeldern und Angeboten zur Teilnahme an besonderen Veranstaltungen. Gerade bei den Veranstaltungen wird dann aber auch erwartet, dass man regelmäßig teilnimmt. Normalerweise sind auch regelmäßige Berichte anzufertigen, in denen man seinen Studienfortgang kommentiert.

Studierende aus dem Ausland
Der DAAD fördert Studierende aus allen Ländern der Welt bei Aus- und Fortbildung sowie Forschungsarbeiten in allen Fachrichtungen. Eignungsvoraussetzung: Abgelegte Zwischenprüfung oder Vordiplom, Deutschkenntnisse. Bewerbung i.d.R. nur im Heimatland beim zuständigen Kultus-/Bildungs- oder Hochschulministerium, in Deutschland beim Akademischen Auslandsamt der zuletzt besuchten Hochschulen (wenn Vordiplom schon in Deutschland gemacht wurde).

Die parteinahen Stiftungen fördern ebenfalls zum Teil Ausländerinnen und Ausländer.

Parteinahe Stiftungen
Alle im Bundestag vertretenen Parteien haben parteinahe Stiftungen gegründet, die auch besonders begabte Studierende, die sich gesellschaftlich engagieren, fördern.

Je nach nahe stehender Partei der Stiftung wird dabei auf unterschiedliche Dinge Wert gelegt. Hier kann euch oft die Hochschule weiterhelfen.

Konfessionelle Träger
Die Förderung der kirchlichen Studienwerke ist an den entsprechenden Glauben gebunden. Auch hier kann die Förderung erst im Studium einsetzen, mit der Bewerbung müssen Gutachten der Hochschule vorgelegt werden.

Wirtschaftsnahe Organisationen
Auch diverse Unternehmen und Wirtschaftsverbände haben Stiftungen oder ähnliches gegründet, die unter bestimmten Umständen auch Studienförderung leisten.

Stipendienprogramm der RUB
Inzwischen hat die Ruhr-Universität ein eigenes Stipendienprogramm, das aktuell 177 Stipendien vergibt. Für unsere Fakultät zählen gute Noten und soziales Engagement als entscheidende Faktoren. Ihr benötigt keine Empfehlung eines Dozenten oder Professors.

Das Stipendium der RUB ist als eines der wenigen Stipendien unabhängig vom eigenen Einkommen oder dem Einkommen der Eltern. Sofern man das Stipendium bekommt, erhält man 300 € pro Monat über einen Zeitraum von einem Jahr.

http://www.ruhr-uni-bochum.de/bildungsfonds/

Sozialbeitrag / Semesterbeitrag
Nicht zu verwechseln mit den (abgeschafften) Studiengebühren, auch wenn es Ähnlichkeiten gibt. Der Sozialbeitrag muss jedes Semester entrichtet werden und bewegt sich zur Zeit in der Größenordnung von 300€. Davon entfallen 182€ auf das Semesterticket, 105€ gehen für Mensa, Wohnheime & Co an das AkaFö und 16€ an den AStA.
Man sollte nicht vergessen ihn rechtzeitig zu überweisen, denn die Mahnung dazu kommt meist in Begleitung einer (vorläufigen!) Exmatrikulations-bescheinigung. Wer sich das ersparen möchte, kann am Lastschriftverfahren teilnehmen, bei dem immer ca. einen Monat vor Semesterbeginn automatisch abgebucht wird.

Krankenversicherung
Jeder Student muss krankenversichert sein, was bei der Einschreibung ja auch kontrolliert wird. Die meisten Studenten sind am Anfang noch über ihre Eltern in einer sog. gesetzlichen Familienversicherung versichert. Aufpassen sollte man jedoch, wenn man bereits berufstätig ist, denn nur bis max. 425€ bzw. 450€ (bei Minijob) pro Monat bleibt dieser Versicherungsschutz erhalten, und auch dann nur bis zu einem Alter von max. 25 Jahren. Am besten mit der eigenen Krankenkasse abklären.

Darüber hinaus kann man als Student auch eine eigene Versicherung zu vergünstigten Konditionen abschließen. Die Höhe der Beitragssätze sind bei den gesetzlichen Versicherung auf ungf.  85€ pro Monat (inkl. Pflegeversicherung) und bei den privaten Versicherern auf ungf. 60-86€ festgelegt. Es existiert also ein Verdienstbereich, in dem ein Plus an Einkommen ein faktisches Minus bedeutet, weil die zu zahlenden 85€ nicht aufgewogen werden.

Wir können nur raten sich hier intensiv schlau zu machen, denn Krankenkassen können sich auch rückwirkend Leistungen rückzahlen lassen!

Mehr Infos siehe Kapitel „Links“
%Wohnheime
Obwohl als Pendler-Uni bekannt, gibt es rund um die RUB ein vielfältiges Angebot an Studierendenwohnheimen. Egal ob ein Zimmer in einer Wohngemeinschaft (WG), ein eigenes Appartement oder ein Einzelzimmer auf einer Gemeinschaftsetage - Studierende der Bochumer Hochschulen sowie der FH Gelsenkirchen können aus einem großen Angebot an hochschulnahem und preisgünstigem Wohnraum auswählen.

Das AKAFÖ bietet sowohl Zimmer als auch Appartements in 18 Wohnheimen an. Alle liegen in der Nähe der Ruhr-Uni oder den anderen Hochschulen in Bochum. Die Kosten betragen „warm“ zwischen 150 Euro und bis zu 490 Euro für eine 3-Raum Wohnung. Zusätzlicher Anreiz ist der Anschluss an das Wohnheimnetz und die Hochgeschwindkeitsverbindung ins Internet. Fairerweise muss man hier allerdings sagen, dass man keine echte „Flatrate“ bekommt.

Das AKAFÖ vergibt auch Einzelzimmer in Großwohngemeinschaften. Hier hat man die Wahl zwischen Zimmer von 12-16m² Größe, die mit einem Waschbecken ausgestattet sind. Bad und Küche teilt man sich allerdings mit 8-12 Leuten von der selben Etage.

Darüberhinaus gibt es aber auch Zimmer in 2-er, 3-er oder 4-er WGs (zB. Die Wohnheime «Studidorf Laerheide» oder «Europahaus»), in denen meistens die Sympathie entscheidet, ob man das Zimmer letztendlich bekommt oder nicht.

Wichtigste Voraussetzung, um ein Zimmer in den vom AKAFÖ verwalteten Gebäuden zu bekommen: Es muss rechtzeitig ein Online-Antrag gestellt werden. Danach heißt es: Geduld haben.

Insider-Tipp: Die netten Sachbearbeiter beim AKAFÖ (zu finden im Studierendenhaus) gelegentlich telefonisch oder mit einem Besuch daran erinnern, dass man auf der Suche ist! Dann kann es sein, dass dein Antrag etwas schneller bearbeitet wird.

Private Wohnheime
Neben den staatlich geförderten AKAFÖ Wohnheimen, gibt es auch einige private Wohnheime, die z.B. von verschiedenen Vereinen, Wohnungsbaugesellschaften oder anderen Förderungswerken verwaltet werden. Hier kann man Zimmer zwischen 150 und 270 Euro mieten, allerdings muss man sich für jedes Wohnheim einzeln bewerben.

Vorsicht ist geboten bei Angeboten von sog. Verbindungen. Hier kann man zwar oft günstig wohnen, muss dafür aber einer solchen Verbindung (oft lebenslang) beitreten und an deren Veranstaltungen teilnehmen, die manchmal recht konservativ erscheinen.

Selber Suchen
Alle die lieber alleine wohnen, mit anderen Leuten eine WG gründen oder in eine bestehende einziehen, finden immer einen Haufen Wohnungsanzeigen, entweder direkt an den schwarzen Brettern in der Uni, im Internet oder z.B im Stadtspiegel.

Da in Bochum, wie auch in den meisten anderen Städten im Ruhrgebiet, kein Wohnraummangel herrscht, gibt es eine Menge bezahlbarer Wohnungen. Bei der Suche sollte man die zusätzlichen Kosten für Telefon und Internet, sowie Heizung, Strom, Wasser und eventuell Gas im Hinterkopf behalten (Nebenkosten schimpft sich das).

Ein-Personen-Wohnungen gibt es außerhalb der Innenstadt oft ab ca. 300 Euro. Wer in eine WG zieht, kann auch zu Preisen wohnen, die ähnlich denen in Wohnheimen sind. Die meisten Inserate findet man übrigens im Internet.

Mehr Infos siehe Kapitel „Links“
%Sport
Für jeden, auf den das Motto „Sport ist Mord“ nicht zutrifft, hat das Angebot des RUB Hochschulsports etwas parat. Die RUB verfügt über mehrere große Sporthallen, welche sich unterhalb der Mensa befinden, sowie Außensportanlagen an der Markstraße. Die Plätze stehen zu bestimmten Zeiten frei zu Verfügung. Dort kann man mit KommilitonInnen oder Studierenden anderer Richtungen gespielt werden.

Neben diesen Sportarten sind im Sportangebot des Hochschulsports auch Kurse mit Trainern im Angebot. Hier kann unter anderem Fechten, Karate und Trampolin springen erlernt werden.

Sehr beliebt sind auch die allgemeinen Fitnesskurse, welche mit Laufen, Krafttraining und Dehnübungen dem Körper Kraft und Ausdauer verleihen sollen. Auch im Wassersport-Bereich ist das Angebot groß. Die RUB verfügt über ein Hallenbad im Uni Center. Dort werden mehrere Schwimmkurse angeboten. Um daran teilzunehmen, muss man am Anfang des Semesters bei der Einteilung in die Schwimmgruppen dabei sein. Im Hallenbad können auch andere Sportarten wie das Unterwasser- Rugby und Tauchen betrieben werden.

Alle angebotenen Kurse sowie Trainingszeiten findet ihr auf der Homepage des Hochschulsports. 

Kultur an der RUB
Boskop (manchmal „boSKop“ geschrieben) ist die „bochumer Studentische Kulturoperative“, des Kulturbüros vom AKAFÖ und damit beauftragt, an den Bochumer Hochschulen studentische Kultur anzuregen und zu fördern. Dazu bietet Boskop eine Vielzahl von interessanten Workshops, internationale Kulturtreffen, musikalische Aufführungen und Themenabende an.

Im Kultur Café direkt auf dem Campus Gelände wird monatlich die Blues Session Bochum angeboten. Dort treten wechselnde Jazz und Blues Bands auf und im Anschluss findet meist ein freies „Jammen“ statt.

Wer Interesse an Internationalen Filmen des Ostens hat, kann den wöchentlich stattfindenden osteuropäischen Film-abend kostenlos besuchen.

Besonders interessant sind die Workshops: Sie laufen in der Regel ein Semester lang. Hier kann man z.B. die Kunst des Cocktailmixens erlernen, sich mit anderen über Literatur unterhalten oder sich im kreativen Schreiben üben. Hier werden auch Sport und Tanzarten aus fremden Ländern wie Capoeira, Tango und orientalischer Tanz gelehrt. Die Anmeldung für die Workshops findet i.d.R. am Anfang des Semesters im Foyer der Mensa statt.

Wer erstmal mit den Standardtänzen anfangen möchte, dem seien die Tanzkurse des AStA ans Herz gelegt.

Daneben gibt es noch viele andere kulturelle Initiativen. Z.B. findet im Sommer das internationale Videofestival statt. Der Studienkreis Film (SKF) bestimmt das wöchentliche Kinoprogramm, welches im HZO 20 gezeigt wird. Der wirklich kostengünstige Besuch im Unikino ist auf jeden Fall lohnenswert. Es ist zu empfehlen, sich ein Kissen mitzubringen!


Kneipen
Das Bermuda-3-Eck!
Das „Bermuda-3-Eck“, wie vor allem die Ecke der Innenstadt rund um den Engelbertbrunnen genannt wird, erfreut sich großer Beliebtheit, und das nicht nur am Abend. Um ein paar der vielen verschiedenen Kneipen kennen zu lernen, empfehlen wir unsere Kneipen-Tour in deiner ersten Uniwoche.

Absinth
Rottstr. 24, 44793 Bochum (Nähe Rotlichtviertel). Urige Kneipe mit buntgemischtem Publikum. Und, wie der Name schon verrät, großer Absinth-Auswahl!

Kultur Café
Größter Vorteil: direkt an der Uni. Perfekt geeignet zum Lernen, gemütlich einen Kaffee trinken, Leute treffen, sowie ein Bier vor, zwischen oder nach den Vorlesungen. Abends gibt es dort auch kulturelle oder politische Veranstaltungen.


Wohnheimkneipen
Hierbei handelt es sich um Kneipen in Wohnheimen für Studierende. Diese werden i.d.R. von den Bewohnern geführt und glänzen nicht nur durch Gemütlichkeit, sondern auch durch gute Preise. Leider öffnen und schließen jedes Jahr ein paar Wohnheimkneipen, so dass wir einfach keinen Überblick mehr darüber haben, welche gerade noch existiert und welche nicht. Fragt einfach rum und haltet die Ohren auf.

Tipp: Häufig haben diese Kneipen nur an bestimmten Wochentagen geöffnet.


Discotheken & Clubs
So, und wenn euch das jetzt immer noch nicht genug ist, hier noch ein paar Tipps zur Wochenend- und Freizeitgestaltung in Bochum:

Matrix Rockpalast (Hauptstr. 200, 44892  Bochum): Gothic bis Punk
Hier werden sehr viele Musikwünsche befriedigt und je nachdem was gerade für ein Special ist, kommt man auch umsonst rein. Dazu werden hier teilweise Konzerte gespielt.
http://www.matrix-bochum.de

Untergrund (Kortumstr. 101, 44787 Bochum): Samstags Alternative, Rock und Indie, Freitags Events und Gemischtwaren
Der Untergrund befindet sich quasi mitten in der Stadt Bochum und der Eingang ist teilweise zu übersehen. Wenn man jedoch erst einmal drin ist und die Stufen nach unten gemeistert hat, erwartet einen eine kleine Tanzfläche. Für nähere Infos hängen auch an der Uni sehr oft Plakate aus, auf denen dann auch die jeweils gespielte Richtung angegeben wird.
http://www.myspace.com/untergrundclub


Schwimmbäder
Aquaris Schwimmbad und Saunaworld (Herner Straße 299, 44809 Bochum):
http://www.aquaris.de

Hallenbad Querenburg „Uni-Bad“ (Hustadtring. 157, 44801 Bochum)
Morgens stark ermäßigt für Studenten

Freizeitbad Heveney (Kemnader See, Querenburger Strasse 35, 58455 Witten): 
http://www.kemnadersee.de



Kinos
Bofimax-Kinocenter (Kortumstr. 51, 44787 Bochum):
http://bofimax.de/

Casablanca Filmtheater (Kortumstr. 11, 44787 Bochum):
http://www.casablanca-bochum.de/

Union Kino (Kortumstr. 16, 44787 Bochum)
http://kino-bochum.de/

UCI Kinowelt (Ruhr Park)
http://www.uci-kinowelt.de/

Studienkreis Film („SKF”, RUB):  Von Studenten für Studenten
http://dbs-lin.ruhr-uni-bochum.de/skf/

Theater
ET CETERA Variete (Herner Str. 299, 44809 Bochum):
http://www.variete-et-cetera.de

Prinz-Regent-Theater (Prinz-Regent-Str. 50 – 60, 44795 Bochum):
http://www.prinzregenttheater.de

Schauspielhaus Bochum (Königsallee 15, 44789 Bochum):
http://www.schauspielhausbochum.de
%Es soll ja Leute geben, die einen Laptop in die Uni schleppen. Wir alle. 

Da ein Computer ohne Internet doch recht langweilig ist, hier ein kurze Zusammenfassung, damit du ins Netz kommst. Dafür hast du 3 Möglichkeiten:

Per HIRN-Port
Mit einem normalen Netzwerkkabel. Einfach in eine mit H.I.R.N. gekennzeichnete Dose einstöpseln und sich automatische eine IP geben lassen.

Wenn du das erste Mal eine Internetseite aufrufst, wirst du auf eine Loginseite umgeleitet. Hier gibst du deine LoginID und dein Passwort ein. Sollte das nicht klappen, kannst du auch login.rz.rub.de manuell aufrufen. Los geht‘s…

Per WLAN und VPN
Du verbindest dich mit dem Accesspoint RUB-WLAN. Natürlich musst du dir deine IP wieder automatisch zuweisen lassen.

Sobald du surfen willst, wirst du auf eine spezielle Seite umgeleitet, denn ohne die spezielle Cisco Zusatzsoftware (VPN) kannst du nur auf manchen internen Seiten surfen. Die Startseite beschreibt für die nötigen Schritte.

Vorteil hier: Du kannst dich mit derselben Software auch von Zuhause in das Uni-Netz einklinken, um so z.B. an bestimmte Dokumente zu gelangen.

Per WLAN mit eduroam
Dies ist die zu bevorzugende Methode. Hierfür benötigst du keine Zusatzsoftware. Außerdem solltest du damit auch an anderen Unis surfen können.

Alle nötigen Schritte findest du hier:
https://noc.rub.de/web/wlan

Hinweis:
Achte auch deine freigegeben Ordner, es soll nämlich tatsächlich Leute auf dem Campus geben, die einfach mal das Netzwerk nach allen Freigaben durchsuchen.

Übersichtsseite über die WLAN Möglichkeiten des Rechenzentrums:

http://www.rz.ruhr-uni-bochum.de/dienste/netze/wlan/

%Tipps & Tricks
	www.fs.ai.rub.de
	forum.ai-rub.de
	www.studis-online.de
	www.rub.de/studienbuero

Einrichtungen 
	www.rub.de
	www.rub.de/studierendensekretariat
	www.ini.rub.de
	www.et-cip.rub.de
	www.rub.de/rz

Studierendenschaft
	www.asta-bochum.de
	www.stupa-bochum.de
	www.fsvkbo.de


Internet & E-Mail
	login.rz.rub.de
	mail.rub.de





Krötenwanderung
	www.akafoe.de
	www.bafög.de
	www.bafoeg-rechner.de
	www.rub.de/bildungsfonds
	www.rub.de/studfinanz

Wohnen in Bochum
	www.akafoe.de/wohnen
	www.bochumer-wohnstaetten.de
	www.wg-gesucht.de
	www.wg-welt.de
	www.allstudents.de
	www.easywg.de

Freizeittipps
	www.hochschulsport-bochum.de
	www.akafoe.de/boskop
	www.bochum.de
	www.bermuda3eck.de

Sonstiges
	www.das-labor.org
	www.sz-bochum.de
	www.fiff.de
	www.protestkomitee.de
	www.bo-alternativ.de
%0
Vorangestellt kennzeichnet die „0“ in den Gebäuden die Etagen unterhalb der Forumsebene. Ein Erdgeschoss, also eine 0. Etage selber, gibt es (außer im UFO) jedoch nicht, die Zählung beginnt oberhalb der Gebäudemitte sofort bei 1.

42
Die Antwort auf die universelle Frage nach dem Leben, dem Universum und allem. Genügt nicht zum Bestehen.

AI
•	Angewandte Informatik — dein Studiengang.
•	Anonyme Informatiker — Selbsthilfegruppe
•	Amnesty International — helfen uns nicht, trotz der vorherrschenden Zustände
•	Artificial Intelligence — Künstliche Intelligenz

Allgemeiner Studierenden-Ausschuss (AStA)
Studentische Interessensvertretung auf Uni-Ebene. Wird vom Studierendenparlament einmal im Jahr gewählt. Den AStA findest du im Studierendenhaus gegenüber der Uni-Verwaltung.

Akademisches Förderungswerk (AKAFÖ)
Verantwortlich für Mensen, Cafeten, staatliche Wohnheime und andere Dinge, die das Studi-Herz begeistern. Zu finden im Studierendenhaus in den Räumen 059, 060 und 056.

Audimax
Auditorium Maximum (lat. “Das größte Publikum”). Der größte Hörsaal der Uni. An der RUB das runde Gebäude in der Mitte, soll eine Muschel darstellen (kein Scherz).

Beurlaubung
Aus welchen Gründen auch immer du dich für ein Semester beurlauben lassen willst, diese Beurlaubung musst du im Uni-Sekretariat beantragen und genehmigen lassen. Die Urlaubssemester werden nicht auf die Studiendauer angerechnet und der Studienplatz bleibt erhalten.

Bochumer Studentische Kulturoperative (Boskop / BoSKop)
Unterstützer und Veranstalter vieler studentischen Kulturaktivitäten (Video, Literatur, Malen, Gestalten, 39 Theater u.s.w.). Sitzt im Wohnheim Sumperkamp 9-15. Anmeldungen für Kurse finden auch dort statt.

Botanischer Garten
Grünanlage im Süden des Unigeländes. Die Frage, ob es auch nicht-botanische Gärten gibt, konnte noch nicht abschließend geklärt werden.

Bundes-Ausbildungsförderungsgesetz (BAföG)
Gesetz, welches regelt, dass du keine oder nur unzureichende Ausbildungsförderung erhältst. Das für dich zuständige BAföG-Amt befindet sich im Uni-Verwaltungsgebäude auf der Eingangsebene. Bei Fragen oder Schwierigkeiten wende dich bitte an die BAföG-Beratung des AStA.

CCC - Chaos Computer Club
1981 gegründeter deutscher Verein, in dem sich Hacker zusammengeschlossen haben und inzwischen ca. 2000 Mitgliedern hat.

Caféte
Länger als die Mensa geöffnete Anlaufstellen für den kleinen Hunger oder Durst nebenbei. Caféten befinden sich verteilt auf dem ganzen Campus.

CIP-Insel
„Computer Investment Program“. Eine Ansammlung von Computern, auch auf dem ganzen Campus.

Credit Point (CP)
Bewertungskriterium für Studienleistungen, ein CP entspricht etwa 30 Arbeitsstunden. 180 braucht man für den Bachelor, 120 für den Master.

Dekan, Dekanat
Der Dekan führt die Geschäfte eines Studienganges und vertritt ihn innerhalb der Hochschule, gegenwärtig Prof. Dr. Laurenz Wiskott.

Deutsches Forschungsnetz (DFN)
Schnelles Backbone-Netz, das unter anderem die Unis verbindet.

Ersti, S. 1 ff.
Wenn du das hier liest, bist du höchstwahrscheinlich einer.

F___
Kennzeichnet in Raumnummern die Flachbauten zwischen den eigentlichen Gebäuden. Der zweite Buchstabe gibt an, ob der Flachbau westlich (W) oder östlich (O) des Gebäudes liegt. ICFW beispielsweise ist der Flachbau, der sich vom Forum aus vor dem Gebäude IC befindet.

Fachschaft (FS)
Zusammenschluss aller Studenten eines Studienganges, in diesem Fall also alle Studenten des Studienganges Angewandte Informatik.

Fachschaftsrat (FSR)
Der auf der Vollversammlung gewählte Fachschaftsrat setzt eure Interessen gegenüber der Uni-Verwaltung und dem AStA durch. Sollte für dich erster Ansprechpartner bei Fragen oder Problemen aller Art sein.

Fachschafts-VertreterInnen-Konferenz (FSVK)
Regelmäßig zusammentretendes Gremium aus Vertretern aller Fachschaften. Koordiniert die Fachschaftsarbeit und entschiedet über Anträge.

Fakultätsrat (Fakrat)
Wird einmal im Jahr (meist im Juni) bei den Gremienwahlen gewählt. Er setzt sich aus acht Professoren, zwei wissenschaftlichen Mitarbeitern, zwei nicht-wissenschaftlichen Mitarbeitern und drei Studenten zusammen. Vorsitzender ist der Dekan. Der Fakultätsrat ist das oberste beschlussfähige Gremium einer Fakultät. Hier finden Verhandlungen über Studienordnungen, Lehrpläne und Berufungen von Professoren statt. Bei uns übernimmt diese Aufgaben der Gemeinsam beschließende Ausschuss.

FlexNow
Online-Tool, mit dem Studenten ihre Prüfungen selbstständig An- und Abmelden können (in der Theorie). Funktioniert nicht mit allen Prüfungen, mit allen Browsern oder bei Vollmond. Kann mit den Rechnern der CIP-Insel genutzt werden.

Forum
•	Die universelle Plattform für den Austausch zwischen allen Studierenden, live von unserem Fachschafts-Server. Hier findet Ihr die brandheißen Informationen als erstes.
https://fs.ai.rub.de/forum
•	Bezeichnung für die Mitte der Uni (also der Platz zwischen Audimax, Universitäts-Bibliothek und dem HZO), der bei der Definition des Forumslevel eine gewaltige Rolle spielt.

Forumslevel
•	Normalnull der Uni, alle Etagenbezeichnungen in den Gebäuden werden relativ zum Forum gerechnet.
•	Noch zu erstellende Map für diverse Spiele, um Uni-Frust abzureagieren.

Fundbüro
Das Fundbüro der Uni ist gleichzeitig der Infopoint im Computerpool im Eingangsbereich der Universitätsverwaltung.

Gemeinsam beschließender Ausschuss (GBA)
Der Gemeinsame beschließende Ausschuss entspricht in der AI dem ->Fakultätsrat

HIRN, HIRN-Port
HochschulInternes RechnerNetz, kommt jeder Student rein, entweder über eine LAN-Dose (HIRN-Port) oder WLan (per eduroam oder RUB-WLAN), wenn man denn Empfang hat. Oh RZ, lass HIRN regnen!

Hochschulrat
Seit 2008 höchstes Gremium der Uni.

IC
Gebäude, wurde wegen PCB-Belastung kernsaniert. Danach konnte man nicht rein wegen PCB-Belastung.

International Office (IO)
Das International Office koordiniert die internationalen Beziehungen der Universität und ist Ansprechpartner für alle Fragen rund um die Internationalität von Lehre und Forschung. 

Institut
Eine kleine Selbstverwaltungseinheit in den Abteilungen / Fakultäten. Gliedert sich meist nach wissenschaftlichen Tätigkeiten.

Java
•	Amerikanischer Slangausdruck für Straßencafe.
•	Nach 1.) benannte Programmiersprache von SUN (mittlerweile Oracle).
•   Inselgruppe, die diesen Namen wohl bald abgeben muss, weil sie die Lizenzgebühren an Oracle nicht mehr zahlen kann.

Kanzler
Der oberste Verwaltungsbeamte der Ruhr-Universität.

Konferenz der Informatik-Fachschaften (KIF)
Informations- und Aktionsplattform für Vertreter aller deutschsprachigen Informatik-Fachschaften. Quelle für lustige Plüschtier-Nähanleitungen.

Matrikelnummer
Ist auf dem Studierendenausweis aufgedruckt und wird beim Ausfüllen vieler Formulare, sowie bei den Klausuren benötigt. Kannst du bald auswendig.

Mensa
Nahrungsaufnahmestätte hinter dem Audimax mit täglich wechselnden Gerichten, Nudeln gibt es immer (an der Nudeltheke). Man kann wählen zwischen zwei Sprintern (Salat im Preis enthalten), zwei Komponentenessen (Beilage gegen Aufpreis) und dem Aktionsmenü (teuer).

N.N.
Abk. (nomen nominandum) wird immer dann verwendet, wenn die ausführende Person noch nicht feststeht.

Prüfungsamt
Verwaltet unsere Prüfungsergebnisse und ist erste Anlaufstelle für Leistungsanerkennung.

Prüfungsausschuss (PA)
Entscheidet über den Ablauf der Prüfungen, setzt Prüfungsordnung fest und erkennt bereits erworbene Prüfungsleistungen an. Für Quereinsteiger also eine wichtige Anlaufstelle. Zudem ist er für alle Arten von Anträgen zuständig.

Prüfungsordnung (PO)
Die vom PA festgelegten Regeln, nach denen Prüfungsleistungen erbracht, gewertet und berechnet werden.

Rechenzentrum (RZ)
Hier gibt es Lizenzen und Hilfe für diejenigen, die ihr Passwort vergessen haben.

Regelstudienzeit
In den Prüfungsordnungen angegebene, sehr optimistische Zeitspanne, in der das Studium absolviert werden soll. Unter anderem orientieren sich die BAföG-Bestimmungen an dieser Zeitspanne.

Rektor
Der Rektor ist der Vertreter der gesamten Uni gegenüber der Öffentlichkeit und dem Ministerium. Seit Dezember 2015 ist Prof. Dr. Axel Schölmerich im Amt.

Rekursion
Siehe ->Rekursion

Rub Internet Connector (rubicon)
Tool mit dem es manchmal möglich ist, auf diverse elektronische Dienste der Uni zuzugreifen (Studienbescheinigung, Semesterticket, VSPL)

Rückmeldung
Ein bürokratischer Akt, der jedes Semester innerhalb einer bestimmten Frist vorgenommen werden muss. Bei Versäumnis: Vorläufige Exmatrikulation

Semesterticket
Preisgünstiges Ticket, das in Verbindung mit dem Studierendenausweis jeweils für ein Semester zur Benutzung von öffentlichen Verkehrsmitteln berechtigt. Ist im Sozialbeitrag enthalten, kann sich jeder ohne Anmeldung im Foyer der Universitätsverwaltung abholen. Ab 19 Uhr und am Wochenende kann eine zweite Person mitgenommen werden.

Semesterwochenstunden (SWS)
Anzahl der Stunden, die im Laufe eines Semesters in jeder Woche auf Lehrveranstaltungen entfallen. Vor- und Nachbearbeitung sind darin nicht enthalten.

Senat
Wird einmal im Jahr bei den Gremienwahlen gewählt. Vorsitzender ist der Rektor. Der Senat war vor dem Hochschulrat das oberste beschlußfassende Gremium der Universität.

Skript
Schriftliche Ausarbeitung von Vorlesungen, werden manchmal von den Lehrstühlen ausgegeben.

Sozialbeitrag
Pro Semester zu leistende Zahlung, mit der verschiedene Dinge wie das Semesterticket und die Mensa finanziert werden, etwa 270€.

Stipendium
Studierende können bei verschiedenen Stiftungen Stipendien beantragen, deren Höchstgrenze meist über denen des BAFöG liegen und nicht an die Regelförderungszeit gebunden sind.

Studiendekanat
Koordiniert Verwaltungsabläufe des Studiengangs. Insbesondere findet hier auch die Studienberatung statt, was für dich am wichtigsten sein dürfte.

Studienkreis Film (SKF)
Einer der ältesten studentischen Filmclubs Deutschlands. Führt regelmäßig sehr günstig Filme in einem Hörsaal der Uni auf.

Studierendenparlament (StuPa)
Verfügt über 35 Sitze und wird einmal jährlich von allen an der Uni eingeschriebenen Studierenden gewählt. Zu seinen wichtigsten Aufgaben gehören die Wahl des AStA und die Genehmigung des Haushaltes. 

U35
Chronisch überlastete Straßenbahn, die UNI und Hauptbahnhof verbindet.

Uni-Sekretariat
Zuständig für Immatrikulation, Exmatrikulation, Rückmeldung, Beurlaubung etc. Du findest es in der Universitätsverwaltung.

Uni-Center
Auf der anderen Seite der Brücke gelegene Einkaufszone mit grimmigen Sicherheitskräften.

Universitäts-Bibliothek (UB)
In der Uni-Bibliothek darf sich jeder Student ohne weitere Anmeldung Bücher ausleihen. Der Studierendenausweis genügt hierzu. Sie ist zu finden in dem großen Gebäude zwischen Studierendenhaus und Audimax.

Vollversammlung (VV)
Der fromme Wunsch, möglichst viele Studenten in einem Raum anzusammeln. Dies geschieht einmal im Semester für die Fachschaft, um den Fachschaftsrat zu wählen und ihm seine Aufgaben zu geben.

VSPL
•	Für uns nicht verbindliches System zur elektronischen Kurs- und Prüfungsanmeldung
•	Noch vor W3L der Größte ProgrammierGAU an der Uni für mehrere Millionen Euro.

Wohnheim
Jeder eingeschriebene Student der Uni kann bei der AKAFÖ-Wohnheimverwaltung einen Antrag auf ein Wohnheimzimmer stellen.
%Fachschaft Angewandte Informatik
Ruhr-Universität Bochum
Gebäude NB, Raum 1/75
44801 Bochum

E-Mail: fsrai@rub.de

Aktuelle Redaktion / Layout
Niklas Heyne
Nicklas Lindemann

Weitere Autoren (seit der 1. Ausgabe)
Christian Mielers
Yannick Schrör
Patrick Tekath
Guido Knips
Jonas Thiel
Stefan Bäcker
Manuel Beelen
Magdalena Broll
Martin Degeling
Patrick Gerk
Olaf Hülscher
Jennifer Jandt
Sanela Kahrica
Hamid Khosrozadeh
Katharina Kohls
Michael Ksoll
Nina Schneider
Sandra Schulze
Rafael Schypula

V.i.S.d.P
Niklas Heyne

Creative Commons
Dieses gesamte Infoheft und dessen Inhalt stehen unter der CC-Lizenz: Namensnennung - Keine kommerzielle Nutzung-Weitergabe unter gleichen Bedingungen 3.0 Deutschland
http://creativecommons.org/licenses/by-nc-sa/3.0/de
%Auf dem Cover dargestellt
Konrad Zuse, Erbauer des ersten funktionsfähigen Digitalrechners.
Vorherige Coverpersönlichkeiten
2006 - Tux
2007 - Grace Hopper
2008 - Charles Babbage
2009 - Ray Tomlinson
2010 - Alan Turing
2011 - Steve Jobs
2012 - Dennis Ritchie
2013 - Tim Berners-Lee
2014 - Edsger W. Dijkstra
2015 - Ken Thompson
2016 - Ada Lovelace
2017 - Konrad Zuse

Bis auf folgende Ausnahmen sind alle Fotos aus dem Archiv der Ruhr-Universtiät Bochum oder der Fachschaft:

Quelle: commons.wikimedia.org
Seite 5: Micthev, „Analog clock displaying 12:14“, GNU Free Documentation License

Quelle: piqs.de
Seite 10: Bodo Stickan , „Patchfeld“, CC-Lizenz (BY 2.0)
Seite 26: Knipsermann, „Krötenwanderung!“, CC-Lizenz (BY 2.0)
Seite 35: danis, „HTML“, CC-Lizenz (BY 2.0)
http://creativecommons.org/licenses/by/2.0/de/deed.de

Quelle: Stadt Bochum
Seiten 29, 30, 31, 33

%\input{chapters/18-Licensing.tex}
\end{document}